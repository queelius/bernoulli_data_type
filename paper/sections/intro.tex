\documentclass[ ../main.tex]{subfiles}
\providecommand{\mainx}{..}
\begin{document}
\section{Introduction}


The primary mechanism by which a value is an \emph{approximation} is given by the random approximate map model.
If a first-order random approximate map of type $X \amapsto{\fprate}{\fnrate} Y$ takes in an exact value $X$ then it maps to a random approximate value $\AT{y}[\fprate][\fnrate]$ of type $Y$.

We denote that the distribution of values over a type $X$ take on random approximations with $\AT{X}$.
The type is still the same, only the \emph{values} are different with respect to some objective standard, e.g., if $\Fun{f}$ maps a value $a$ to $b$, then ...

A \emph{type} is a set and the elements of the set are called the \emph{values} of the type.

These \emph{values} are approximate values if, according to some objective function they should be $x$ but take on a range of possible values according to the random approximate value model.


An \emph{abstract data type} is a type and a set of operations on 
values of the type.
For example, the \emph{integer} abstract data type is defined by the set of integers and standard operations like addition and subtraction.
A \emph{data structure} is a particular way of organizing data and may implement one or more abstract data types.


\end{document}