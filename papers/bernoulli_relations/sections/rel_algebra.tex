\documentclass[ ../main.tex]{subfiles}
\providecommand{\mainx}{..}
\begin{document}

\section{Approximate relational algebra}
When we replace relations by approximate relations, the relational operations become approximate.

When we directly replace relational operations by approximate relational operations, e.g., a projection creates an approximation of the real projection, this also creates approximate relations.

\subsection{Random approximate composition}
Compose should return a function that is the composition of a list of functions of arbitrary length. Each function is called on the return value of the function that follows. You can think of compose as moving right to left through its arguments.


\begin{theorem}[Composition]
	Consider two approximate relations, $\ASet{R}[1] \subseteq \Set{X}[1] \times \cdots \times \Set{X}[n]$ and $\ASet{R}[2] \subseteq \Set{X}[k] \times \cdots \times \Set{X}[n+p]$, $0 k \leq n$ and $p \geq 0$, respectively with false positive rates $\fprate_1$ and $\fprate_2$ and false negative rates $\fnrate_1$ and $\fnrate_2$. Then, the \emph{join} $\Join(\ASet{R}[1],\ASet{R}[2]) \subseteq \Set{X}[1] \times \cdots \times \Set{X}[n+p]$ is an approximate relation with a false positive rate $\fprate_1 \fprate_2$ and false negative rate $\fnrate_1 \fnrate_2$.
\end{theorem}
\begin{example}
	Let $\APFun{f} \colon \Set{X} \mapsto \Set{Y}$ be an approximate map with a false positive rate $\fprate_1$ and false negative rate $\fnrate_1$ and let $\APFun{g} \colon \Set{Y} \mapsto \Set{Z}$ be an approximate map with a false positive rate $\fprate_2$ and false negative rate $\fnrate_2$.
	
	Then, the composition $\APFun{g} \circ \APFun{g} \colon \Set{X} \mapsto \mathbb{Z}$ is an approximate map with a false positive rate ... and false negative rate...
\end{example}




\end{document}