\documentclass[ ../main.tex]{subfiles}
\providecommand{\mainx}{..}
\begin{document}
\section{Relational algebra}
\label{sec:relations}
integrate:::
Suppose we have a partial function $\Fun{f} \colon X_1 \times \cdots \times X_m \pfun Y_1 \times \cdots \times Y_n$.
This is a special kind of relation where the projection $\Proj$
Experience shows that it is awkward to deal with multi-valued functions and that it is best to treat them as relations (or to change the output domain to be a power set, which is equivalent to view the function as a relation).
:::



The concept of a \emph{relation} is dependent upon the concept of a \emph{set}.
\begin{definition}
A set is an unordered collection of distinct elements from a universe of elements.
\end{definition}
A countable set is a \emph{finite set} or a \emph{countably infinite set}. A \emph{finite set} has a finite number of elements. For example,
\[
    \Set{S} = \{ 1, 3, 5 \}
\]
is a finite set with three elements. A \emph{countably infinite set} can be put in one-to-one correspondence with the set of natural numbers.

The cardinality of a set $\Set{S}$ is a measure of the number of elements in the set, denoted by
\begin{equation}
    \Card{\Set{S}}\,.
\end{equation}
The cardinality of a \emph{finite set} is a non-negative integer and counts the number of elements in the set, e.g.,
\[
    \Card{\left\{ 1, 3, 5\right\}} = 3\,.
\]

Given two sets $\Set{A}$ and $\Set{B}$, a basic construction in set theory is the formation of an ordered pair $\Pair{a}{b}$, where $a \in \Set{A}$ and $b \in \Set{B}$, and whose main property is that $\Pair{a}{b} = \Pair{c}{d}$ \emph{if and only if} $a = c$ and $b = d$.\footnote{This is a non-commutative property, i.e., $\Pair{a}{b}$ does not necessarily equal $\Pair{b}{a}$.} A \emph{tuple} is a generalization of pairs which can consist of an arbitrary number of elements, e.g., $\Tuple{x_1,x_2,\ldots,x_n}$.

\begin{definition}[Cartesian product]
Let $\Set{X}[1], \ldots, \Set{X}[n]$ be $n$ sets. The $n$-fold Cartesian product of $\Set{X}[1],\ldots,\Set{X}[n]$ is given by $\SetBuilder{\Tuple{x_1,\ldots,x_n}}{x_1 \in \Set{X}[1] \land \cdots \land x_n \in \Set{X}[n]}$ and is denoted by the cross product $\Set{X}[1] \times \cdots \times \Set{X}[n]$.
\end{definition}

We are now ready to define the concept of a relation.
\begin{definition}[Relation]
Let $\Set{X}[1],\ldots,\Set{X}[n]$ be sets. A \emph{relation} is any subset of $\Set{X}[1] \times \cdots \times \Set{X}[n]$.
\end{definition}
%In a relation $\Relation{R} \subseteq \Set{X}[1] \times \cdots \times \Set{X}[n]$, the domain of the relation is each $\Set{X}[j]$, $1 \leq j %\leq n$, a domain of the relation. 
We denote that a particular tuple $\Tuple{x_1,\ldots,x_n}$ is in a relation $\Relation{R}$ using either notation $\Relation{R}(x_1,\ldots,x_n)$ or $\Tuple{x_1,\ldots,x_n} \in \Set{R}$. If the relation is \emph{binary}, then may denote that two elements $x$ and $y$ are related by $x \Relation{R} y$.

The set of possible relations over $\Set{X}[1],\ldots,\Set{X}[n]$ is the \emph{power set} $\PS{\Set{X}[1] \times \cdots \times \Set{X}[n]}$, which has a cardinality of $2^{\Card{\Set{X}[1]} \cdots \Card{\Set{X}[n]}}$. As a \emph{base case}, we have \emph{unitary} relations, e.g., $\Relation{R} \subseteq \Set{X}$.

Suppose we have a relation $\Relation{R} \subseteq \Set{X} \times \Set{Y} \times \Set{Z}$.
The \emph{domain} (or type) of the relation is $\Set{X} \times \Set{Y} \times \Set{Z}$.
We identify the component of a particular tuple in the relation by its \emph{index} position, e.g., the $\nth{2}$ component of $\Tuple{x,y,z}$ is $y$ whose \emph{domain} is $\Set{Y}$.

Tables are a convenient way to represent relations. When a relation $\Relation{R}$ is viewed as a table, there are several important observations to keep in mind:
\begin{enumerate}
\item Each row of the table represents a tuple of $\Relation{R}$.
\item The row order is unimportant.\footnote{That is, given a table representing $\Relation{R}$, if we exchange any two rows the table still represents $\Relation{R}$.}
\item The multipicity of a row is unimportant.\footnote{That is, given a table representing $\Relation{R}$, if we duplicate any row the table still represents $\Relation{R}$.}
\item The column order is important.\footnote{The set of values in the $j$-th column correspond to the set of values of the $j$-th component of each tuple.}
\end{enumerate}

\subsection{Relational operations}
Relations are sets, therefore relations over identical domains may participate in operations on sets, like unions and differences.

Moreover, due to the structure of relations as subsets of Cartesian products over domains (sets), additional operations can meaningfully be defined.

First, we revisit the Cartesian product. The Cartesian product is associative and commutative \emph{up to isomorphism} only. Thus, when we take the Cartesian product of two relations, we are pairing every \emph{tuple} in one with every \emph{tuple} in the other.



The \emph{join} operator\footnote{Sometimes denoted the \emph{natural join}.} in relational algebra is the counter-part to the \emph{logical-and} operator in Boolean algebra.
\begin{definition}
Suppose $\Relation{A} \subseteq \Set{X}[1] \times \cdots \Set{X}[n]$ and $\Relation{B} \subseteq \Set{X}[k] \times \cdots \times \Set{X}[n+p]$, where $0 < k \leq n$ and $p \geq 0$.\footnote{The order of the domains in a relation may be changed so that the common domains are in the right place.} The \emph{join} of $\Relation{A}$ and $\Relation{B}$ is the relation defined by
\begin{equation}
    \Relation{A} \Join \Relation{B} = \SetBuilder{ \Tuple{x_1,\ldots,x_{n+p}} }{ \Tuple{x_1,\ldots,x_n} \in \Relation{A} \land \Tuple{x_k,\ldots,x_{n+p}} \in \Relation{B}}\,.
\end{equation}
\end{definition}

The \emph{projection} operator decreases the dimensionality of the relation, i.e., dropping columns from the relation.
\begin{definition}
The projection of $\Relation{R} \subseteq \Set{X}[1] \times \cdots \times \Set{X}[n]$ onto $\Set{X}[j_1] \times \cdots \times \Set{X}[j_k]$ where $j_i \leq n$ for $i=1,\ldots,k$, is defined by
\begin{equation}
   \Proj_{\Set{X}[j_1] \times \cdots \times \Set{X}[j_k]}\!(\Relation{R}) = \SetBuilder
   {\Tuple{x_{j_1},\ldots,x_{j_k}}}{\Relation{R}(x_1,\ldots,x_n)}\,.
\end{equation}
\end{definition}

The \emph{composition} operator is a function of a join and a projection.
\begin{definition}
The composition of $\Relation{A} \subseteq \Set{X}[1] \times \cdots \times \Set{X}[n]$ and $\Relation{B} \subseteq \Set{X}[k] \times \cdots \times \Set{X}[n+p]$ is defined by
\begin{equation}
    \Relation{A} \circ \Relation{B} = \Proj_{\Set{X}[1] \times \cdots \Set{X}[k-1] \times \Set{X}[n+1] \times \cdots \times \Set{X}[n+p]}\left(
        \Relation{A} \Join \Relation{B}
    \right)\,.
\end{equation}
\end{definition}

We illustrate the above results with the following example.
\begin{example}
Consider the relations $\Relation{A}$ and $\Relation{B}$ given respectively by \cref{tbl:ex_r1,tbl:ex_r2}. The \emph{join} of relations $\Relation{A}$ and $\Relation{B}$ is given by \cref{tbl:ex_join}. The projection of $\Relation{A} \Join \Relation{B}$ onto $\Set{X}[1] \times \Set{X}[2] \times \Set{X}[5]$ is given by \cref{tbl:ex_proj}.

\begin{table}[h]
\centering
\caption{Relations}
\begin{subtable}[h]{.4\linewidth}
\centering
\begin{tabular}{llll}
\toprule
$\Set{X}[1]$ & $\Set{X}[2]$ & $\Set{X}[3]$ & $\Set{X}[4]$\\
\midrule
$a$ & $b$ & $a$ & $c$\\
$a$ & $a$ & $a$ & $a$\\
$a$ & $c$ & $c$ & $d$\\
\bottomrule
\end{tabular}
\caption{$\Relation{A}$}
\label{tbl:ex_r1}
\end{subtable}
\begin{subtable}[h]{.4\linewidth}
\centering
\begin{tabular}{lll} 
\toprule
$\Set{X}[3]$ & $\Set{X}[4]$ & $\Set{X}[5]$\\
\midrule
$a$ & $c$ & $a$\\
$c$ & $d$ & $b$\\
$e$ & $f$ & $c$\\
$c$ & $d$ & $d$\\
\bottomrule
\end{tabular}
\caption{$\Relation{B}$}
\label{tbl:ex_r2}
\end{subtable}\\
\begin{subtable}[h]{.4\linewidth}
\centering
\begin{tabular}{lllll}
\toprule
$\Set{X}[1]$ & $\Set{X}[2]$ & $\Set{X}[3]$ & $\Set{X}[4]$ & $\Set{X}[5]$\\
\midrule
$a$ & $b$ & $a$ & $c$ & $a$\\
$a$ & $c$ & $c$ & $d$ & $b$\\
$a$ & $c$ & $c$ & $d$ & $d$\\
\bottomrule
\end{tabular}
\caption{$\Relation{A} \Join \Relation{B}$}
\label{tbl:ex_join}
\end{subtable}
\begin{subtable}[h]{.4\linewidth}
\centering
\begin{tabular}{lll}
\toprule
$\Set{X}[1]$ & $\Set{X}[2]$ & $\Set{X}[5]$\\
\midrule
$a$ & $b$ & $a$\\
$a$ & $c$ & $b$\\
$a$ & $c$ & $d$\\
\bottomrule
\end{tabular}
\caption{$\Proj_{\Set{X}[1] \times \Set{X}[2] \times \Set{X}[5]}(\Relation{A} \Join \Relation{B})$.}
\label{tbl:ex_proj}
\end{subtable}
\end{table}
\end{example}
\end{document}