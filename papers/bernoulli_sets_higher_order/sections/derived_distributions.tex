\documentclass[ ../main.tex]{subfiles}
\providecommand{\mainx}{..}
\begin{document}
\section{Probability distributions of parameters for \emph{first-order} approximations}
\label{sec:characteristics}
The (first-order) random approximate sets are \emph{parameterized} by the \emph{expected} rates of two types of error, false negative and false positive rates.
In this section, we derive the distribution for these rates.

A random variable $\RV{W} \colon \Sigma \mapsto \PlainSet{Y}$ is a function that maps outcomes in the $\sigma$-algebra to a measurable space 
$\PlainSet{Y}$.
The probability that $\RV{W}$ realizes some measureable subset $\PlainSet{Z} \subseteq \PlainSet{Y}$ is given by $\Prob{\RV{W} \in 	\PlainSet{S}} = \Prob{\{w \Given \RV{W}(w) \in \PlainSet{Z}\}}$.

The number of false positives is a random variable given by the following theorem..
\begin{theorem}
\label{thm:fpbinom}
Given $\n$ negatives, the number of \emph{false positives} in an approximate set with a false positive rate $\fprate$ is a random variable denoted by $\FP_\n$ with a distribution given by
\begin{equation}
    \FP_\n \sim \bindist\!\left(\n, \fprate\right)\,.
\end{equation}
\end{theorem}
\begin{proof}
By \cref{asm:fprate}, the uncertain outcome that a negative element \emph{tests} as positive is a Bernoulli trial with a mean $\fprate$.
Since there are $\n$ such independent and identically distributed trials, the number of false positives is binomially distributed with a mean $\n \fprate$.
\end{proof}

The false positive rate $\fprate$ is an \emph{expectation}.
However, the false positive rate of a random approximate set $\ASet{S}$ parameterized by $\fprate$ is \emph{uncertain}.
\begin{theorem}
\label{thm:fpr}
The false positive rate is the random variable, denoted by $\FPR_\n$, defined as
\begin{equation}
    \FPR_\n = \frac{\FP_\n}{\n}\,,
\end{equation}
with an expectation $\fprate$, variance $\fprate(1-\fprate) / \n$, and probability mass function
\begin{equation}
	\Pdf_{\FPR_\n}\!\FuncArgs*{\fprateob \Given \fprate} = \Pdf_{\FP_\n}\!\FuncArgs*{\fprateob \n \Given \fprate}\,.
\end{equation}
over the support $\SetBuilder{ \frac{j}{\n} \in \BlackboardSet{Q}}{j \in \{0,\ldots,\n\}}$.
\end{theorem}
\begin{proof}
By \cref{def:fprate}, the false positive rate is given by the ratio of the number of false positives to the total number of negatives.
By \cref{thm:fpbinom}, given that there are $\n$ negatives, the number of false positives is a random variable denoted by $\FP_\n$.
Therefore, the false positive rate, as a function of $\FP_\n$, is the random variable $\frac{\FP_\n}{\n}$.
The \emph{expected} false positive rate is
\begin{equation}
    \Expect{\frac{\FP_\n}{\n}} = \frac{1}{\n}\Expect{\FP_\n} = \fprate
\end{equation}
and its variance is
\begin{equation}
    \Var{\frac{\FP_\n}{\n}} = \frac{1}{\n^2}\Var{\FP_\n} = \frac{\fprate(1-\fprate)}{\n}\,.
\end{equation}
Finally, $\FPR_\n = \FP_\n / \n$ is a \emph{scaled} transformation of the binomial distribution.
Thus, since $\FP_\n = \n \FPR_\n$,
\begin{equation}
	\PDF{\fprateob_\n \Given \fprate}[\FPR_\n] = \PDF{\n \fprateob}[\FP_\n]\,.
\end{equation}
\end{proof}
The following corollary immediately follows.
\begin{corollary}
	\label{cor:tnbinom}
	Given $\n$ negatives, the number of \emph{true negatives} in a random approximate set with a false positive rate $\fprate$ is a random variable denoted by $\TN_\n$ with a distribution given by
	\begin{equation}
	\TN_\n = \n - \FP_\n \sim \bindist\!\left(\n, 1-\fprate\right)\,.
	\end{equation}
	By definition, the \emph{true negative rate} $\TNR_\n = \TN_\n / \n = 1 - \FPR_\n$.
\end{corollary}

By \cref{thm:fpr}, the more negatives there are, the lower the variance.
\begin{corollary}
\label{cor:fpr_as_vareps}
	Given \emph{countably infinite} negatives, a random approximate set with a false positive rate $\fprate$ is \emph{certain} to obtain $\fprate$.
\end{corollary}
\begin{proof}
We know that the \emph{expected} value for each of the random variables in this sequence is $\fprate$ and the variance is $\fprate(1-\fprate)/\n$.
Immediately, we see that as $\n$ increases, the distribution of false positives must become more concentrated around $\fprate$.
As $\n \to \infty$, the variance goes to $0$, i.e., the distribution becomes degenerate with all of the probability mass assigned to the mean. See \cref{app:cor_fpr_as_vareps} for a more rigorous proof.
\end{proof}

The fewer negatives, the greater the variance.
The maximum possible variance, when $\n=1$ and $\fprate = 0.5$, is $0.25$, may be used as the most \emph{pessimistic} estimate given a situation where we have no information about the false positive rate $\fprate$ and the cardinality of the universal set.

A degenerate case is given by letting $\n = 0$, corresponding to a random approximate set of the universal set which has no negative elements that can be tested.
Respectively, only random \emph{negative} or \emph{positive} approximate sets may be generated for the universal set or empty set.

The number of false negatives is given by the following theorem.
\begin{theorem}
\label{thm:fnbinom}
Given $\p$ positives, the uncertain number of \emph{false negatives} in random approximate sets with a false negative rate $\fnrate$ is modeled as a binomial distributed random variable denoted by $\FP_\p$,
\begin{equation}
	\FN_\p \sim \bindist(\p, \fnrate)\,.
\end{equation}
\end{theorem}
\begin{proof}
By \cref{asm:fprate}, the probability that a positive element \emph{tests} as negative is $\fnrate$.
Thus, each test is a Bernoulli trial.
Since there are $\p = \Card{\Set{S}}$ such independent and identically distributed trials with a probability of ``success'' $\fnrate$, the number of false negatives is binomially distributed.
\end{proof}

The false negative rate $\fnrate$ is an \emph{expectation}.
However, the false false negative rate of an approximate set $\ASet{S}$ parameterized by $\fnrate$ is \emph{uncertain}.
\begin{theorem}
\label{thm:fnr}
The false negative rate realizes an uncertain value as given by
\begin{equation}
    \FNR_\p = \frac{\FN_\p}{\p}
\end{equation}
with a support $\SetBuilder{j / \n}{j = 0,\ldots,\p}$, an expectation 
$\fnrate$, 
and a variance $\fnrate(1-\fnrate) / \p$.
\end{theorem}
The proof follows the same logic as the proof for \cref{thm:fpr}, except we 
replace \emph{negatives} with \emph{positives}.

In \cref{sec:set_theory}, we consider set-theoretic operations like 
\emph{complements}. The \emph{complement} operator applied to an approximate 
set 
of a set with countably infinite negatives is an approximate set of a set with 
countably infinite positives.
\begin{corollary}
An approximate set of a set with countably infinite positives has a false 
negative rate that is \emph{certain} to obtain $\fnrate$.
\end{corollary}
The proof follows the same logic as the proof for \cref{cor:fpr_as_vareps}, 
except we replace \emph{negatives} with \emph{positives}.

The number of true positives is given by the following corollary.
\begin{corollary}
\label{cor:tpbinom}
Given $\p$ positives, the number of \emph{true positives} in an approximate set 
with a false negative rate $\fnrate$ is a random variable denoted by $\TP_\p$
with a distribution given by 
\begin{equation}
    \TP_\p \sim \bindist(\p, \tprate)\,.
\end{equation}
By definition, the \emph{true positive rate} is given by $\TPR_\p = 1 - 
\FNR_\p$.
\end{corollary}
The proof follows the same logic as the proof for \cref{thm:fpr}.

Many other properties of random approximate sets follow from these distributions.
For instance,
\begin{equation}
	\Card{\ASet{A}[\fprate][\tprate]} = \TP_\p + \FP_\n\,,
\end{equation}
which has an expectation of $\n \fprate + \p \tprate$ and variance of $\n \fprate(1-\fprate) + \p \tprate(1-\tprate)$, which is the generalization of the binomial distribution known as the \emph{Poisson binomial distribution}.

If we do not know $\p$, the cardinality $\Set{A}$, but have observed $\ASet{A} = \Set{B}$, then $\Set{B}$ has a cardinality that tends to be centered around $u \fprate + \p (\tprate - \fprate)$ where $u$ is the cardinality of the universal set.
Solving for $\p$ yields a method of moments estimator
\begin{equation}
	\widehat{\p} = \frac{\Card{\Set{B}} - u \fprate}{\tprate - \fprate}\,.
\end{equation}
If the universal set $\Set{U}$ is infinite, then this estimator is undefined.

\subsection{Asymptotic limits}
\label{sec:asymtotic}
The false positive and false negative rates are a function of the cardinality of the objective and universal sets.
The limiting distributions for the false positive and true positive rates are given by the following theorems.
\begin{theorem}
    \label{thm:approxfpr}
    By \cref{thm:fpr}, the uncertain false positive rate $\FPR_\n$ converges in
    distribution to the normal distribution with a mean $\fprate$ and a 
    variance $\fprate(1-\fprate)/\n$, written
    \begin{equation}
    \label{eq:approxfpr}
	    \FPR_\n \xrightarrow{d} \normdist\!\left(\fprate, \fprate(1-\fprate) / \n\right)\,.
    \end{equation}
    Similarly, by \cref{cor:tpr}, the uncertain true positive rate of an approximate  set of $\p$ positives, denoted by $\TPR_\p$, converges in distribution to the normal distribution with a mean $\tprate$ and a variance $\tprate(1-\tprate)/\p$, written
    \begin{equation}
    \label{eq:approxtpr}
    	\TPR_\p \xrightarrow{d} \normdist\!\left(\tprate, \tprate(1-\tprate) / \p\right)\,.
    \end{equation}
\end{theorem}
\begin{proof}
    By \cref{eq:proof_fpr_as_vareps} in the proof of \cref{cor:fpr_as_vareps}, 
    given 
    $\n$ negatives, the false positive rate is
    \begin{equation}
    	\FPR_\n = \frac{\RV{X_1}}{\n} + \cdots + \frac{\RV{X_\n}}{\n}\,,
    \end{equation}
    where $\RV{X_1},\ldots,\RV{X_n}$ are $n$ independent Bernoulli trials each with a mean $\fprate$ and a variance $\fprate(1-\fprate)$.
    Therefore, by the central limit theorem, $\FPR_\n$ converges in distribution to a normal distribution with a mean $\fprate$  and a variance $\fprate(1-\fprate)/\n$.
    The proof for the true positive rate follows the same logic.
\end{proof}
By \cref{eq:approxfpr,eq:approxtpr},
\begin{equation}
	\TNR_\n \xrightarrow{d} \normdist\!\left(1-\fprate, \fprate(1-\fprate) / \n\right) \text{ and } \FNR_\n \xrightarrow{d} \normdist\!\left(1-\tprate, \tprate(1-\tprate) / \p\right)\,.	
\end{equation}

The random approximate set model is the \emph{maximum entropy} probability distribution for the indicated false positive and true positive rates, e.g., any estimated $\alpha$-confidence intervals are the largest intervals possible for the indicated $\alpha$ and therefore represent a worst-case uncertainty.

If we generate an approximate set, the uncertain false positive and true positive rates realize certain values, i.e., $\FPR_\n = \fprateob$ and $\TPR_\p = \tprateob$.
If the sample space is countably infinite, the distribution is degenerate, e.g., $\FPR_\n = \fprate$ with probability $1$. 
However, for finite sample spaces, the outcomes are uncertain.
If these outcomes can be \emph{observed}, e.g., it is not too costly to compute, the exact values $\fprateob$ and $\tprateob$ may be recorded.
If these outcomes cannot be observed, e.g., it is too costly to compute or the information to compute $\fprateob$ or $\tprateob$ is not available, we may use the probabilistic model to inform us about the distribution of false positive rates.

\emph{Confidence intervals} that contain the true false positive rate $\fprateob$ and the true true positive rate $\tprateob$ are given by the following corollaries.
\begin{theorem}
    Given a random approximate set parameterized by $\fprate$ and $\tprate$, asymptotic $\alpha \cdot 100\%$ confidence intervals for the false positive rate and true positive rate are respectively
    \begin{equation}
    \label{eq:conf_fpr}
    \fprate \pm \sqrt{\frac{\fprate(1-\fprate)}{\n}} \Phi^{-1}(\alpha/2)
    \end{equation}
    and
    \begin{equation}
    \label{eq:conf_tpr}
    \tprate \pm \sqrt{\frac{\tprate(1-\tprate)}{\p}} \Phi^{-1}(\alpha/2)\,,
    \end{equation}
    where $\Phi^{-1} \colon [0,1] \mapsto \RealSet$ is the inverse cumulative distribution function of the standard normal.
\end{theorem}
As a worst-case (maximum uncertainty), we may let $\n = \p = 1$ in \cref{eq:conf_fpr,eq:conf_tpr}.
%However, if the universe is large and the objective sets are relatively small, a slightly optimistic confidence interval for $\fprate$ is provided by letting $\n = \Card{\Set{U}}$.


\end{document}