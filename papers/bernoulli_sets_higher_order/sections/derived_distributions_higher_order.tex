\documentclass[ ../main.tex]{subfiles}
\providecommand{\mainx}{..}
\begin{document}
\section{Probability distributions of parameters for \emph{higher-order} approximations}
\label{sec:characteristics2}

%Higher-order random approximate sets are less trivial, and a full probabilistic model, while tractable, is generally not as useful.

The false negative rate is a random variable that is a mixture? of binomially distributed random variables and likewise for the false positive rate.
We characterize their error rates by their expected values and variance and show their normal approximation when the sampling distribution of these rates are important.

\subsection{Compositions of random approximate sets}
\label{sec:set_theory}
The union of two first-order random approximate sets is given by the following theorem.
\begin{theorem}
	%\begin{equation}
	%	\Prob{\SetIndicator{\ASet{A}}(a) \cap \SetIndicator{\ASet{A}}(b) \Given \SetIndicator{\Set{A}}(a) \cap \SetIndicator{\Set{A}}(b)} = 
	%	\Prob{\SetIndicator{\ASet{A}}(a) \Given \SetIndicator{\Set{A}}(a)} \Prob{\SetIndicator{\ASet{A}}(b) \Given \SetIndicator{\Set{A}}(b)}
	%\end{equation}
	The union of first-order random approximate sets $\ASetFO{A}[\epsilon_1]$ and $\ASetFO{B}[\epsilon_2]$ is up to a \emph{third-order} random approximate set with a random error rate given by
	\begin{equation}
		\Delta = \sum_{i=1}^{3} \alpha_j \RV{X_j}
	\end{equation}
	with an expectation 
	\begin{equation}
		\epsilon = \alpha_1 (1-\epsilon_1)\epsilon_2 + \alpha_2 (1-\epsilon_2)\epsilon_1 + (1-\alpha_1 - \alpha_2 - \alpha_3)\epsilon_1 \epsilon_2\,,
	\end{equation}
	a \emph{false negative rate}
	\begin{equation}
		\fnrate = \epsilon_1 \epsilon_2
	\end{equation}
	and a \emph{false positive rate}
	\begin{equation}
		\fprate = \epsilon_1 \epsilon_2\,,
	\end{equation}
	where $\alpha_1 \coloneqq \frac{\Card{\Set{A} \SetDiff \Set{B}}}{\Card{\Set{U}}}$, $\alpha_2 \coloneqq \frac{\Card{\Set{B} \SetDiff \Set{A}}}{\Card{\Set{U}}}$, and $\alpha_3 \coloneqq \frac{\Card{\Set{A} \SetIntersection \Set{B}}}{\Card{\Set{U}}}$.
\end{theorem}






The set of $n$-arity operations on set $\Set{X}$ is given by $\Set{X}^n \mapsto \Set{X}$.
In this section, we consider binary operations on $\PS{\Set{U}}$, like $\SetUnion \colon \PS{\Set{U}} \times \PS{\Set{U}} \mapsto \PS{\Set{U}}$, and the result of providing \emph{random approximate sets} as input.

%Even though we are generating random approximate sets from random approximate sets, we do not consider the compositions in this section to be higher-order approximations since we have zero control over the distribution of the parameters, like the false positive rate.
%Instead, such characteristics are a function of the random approximate sets being composed.

Given the Boolean algebra
$\left(\PS{\Set{U}}, \SetIntersection, \SetUnion, \SetComplement, \EmptySet, \Set{U}\right)
$
we derive the random approximate sets that result from the union (join) or complement of random approximate sets.
Since unions and complements form a \emph{complete basis} for Boolean algebras, we may express other Boolean operations as a composition of these two operations, e.g., $\SetDiff[\ASet{A}][\ASet{B}] = \SetComplement[\SetUnion[\SetComplement[\ASet{A}]][\ASet{B}]]$.

The random approximate sets that result from union operations on random approximate sets are given by the following theorems.
\begin{theorem}
	\label{thm:union_fp}
	The union of two random approximate sets respectively with true negative rates $\tnrate_1$ and $\tnrate_2$ is a random approximate set with a true negative rate 
	$\tnrate_1 \tnrate_2$.
\end{theorem}
\begin{proof}
	Suppose we have two sets $\Set{A}$ and $\Set{B}$ with false positive rates $\fprate_1$ and $\fprate_2$. The false positive rate $\fprate$ of $\SetUnion[\ASet{A}][\ASet{B}]$ is a probability conditioned on a negative for $\SetUnion[\Set{A}][\Set{B}]$ being a positive for $\SetUnion[\ASet{A}][\ASet{B}]$.
	
	Switching to the Boolean vector representation, suppose we randomly select an element from the universe, denoted by $x_j$, such that $\neg \RV{A}_j \lor \neg \RV{B}_j$ is true.
	
	The expected false positive rate of the union is defined by the probability
	\begin{equation}
	\fprate = \Prob{\AVecComp{A} \cup \AVecComp{B} \Given B_1 \cap B_2}\,.
	\end{equation}
	By DeMorgan's law, the union of sets is the complement of the intersection of their complements. That is,
	\begin{equation}
	\label{eq:proof_union_fp_1}
	A_1 \cup A_2 \equiv \left(A_1' \cap A_2'\right)'
	\end{equation}
	and thus
	\begin{equation}
	\label{eq:proof_union_fp_2}
	\fprate = \Prob{\left(A_1' \cap A_2'\right)' \Given B_1 \cap B_2}\,.
	\end{equation}
	Since either an event or the \emph{complement} of the event is certain to occur, $\Prob{E} + \Prob{E'} = 1$, the above equation may be rewritten as
	\begin{equation}
	\label{eq:proof_union_fp_3}
	\fprate = 1 - \Prob{A_1' \cap A_2' \Given B_1 \cap B_2}\,.
	\end{equation}
	Since $A_1'$ and $A_2'$ are independent,
	\begin{equation}
	\fprate = 1 - \Prob{A_1' \Given B_1 \cap B_2} \Prob{A_2' \Given B_1 \cap 
		B_2}\,.
	\end{equation}
	Since $A_1$ is conditionally independent of $B_2$ and $A_2$ is conditionally independent of $B_1$, we may rewrite the above equation as
	\begin{equation}
	\fprate = 1 - \Prob{A_1' \Given B_1} \Prob{A_2' \Given B_2}\,.
	\end{equation}
	$A_j$ denotes $\RV{X} \in \ASet{S}[j]$, therefore $A_j'$ denotes $\RV{X} \notin \ASet{S}[j]$. Substituting the definition of $A_1'$, $A_2'$, $B_1$, and $B_2$ into the above equation gives
	\begin{equation}
	\fprate = 1 - \Prob{\RV{X} \in \ASet{A} \Given \RV{X} \notin \Set{A}} 
	\Prob{\RV{X} \notin \ASet{B} \Given \RV{X} \notin \Set{B}}\,.
	\end{equation}
	By definition, $\Prob{\RV{X} \notin \ASet{A} \Given \RV{X} \notin 
		\Set{A}}$ is the true negative rate $\tnrate_1$ and likewise for $\ASet{B}$. Thus,
	\begin{equation}
	\fprate = 1 - \tnrate_1\tnrate_2\,.
	\end{equation}
\end{proof}

The limiting probability distribution of the uncertain true negative rate of the union of $\ASet{A}$ and $\ASet{B}$ is thus
\begin{equation}
\TNR_\n \sim \normdist\!\left(\tnrate_1 \tnrate_2, \frac{\tnrate_1 \tnrate_2 (1-\tnrate_1 \tnrate_2)}{\n}\right)
\end{equation}
where the number of negatives $\n = u - \Card{\SetUnion[\Set{A}][\Set{B}]} \leq u$, which is a value between $0$ and $u$. Since this is a limiting distribution, presumably $\n$ is large, and as $\n \to \infty$ the distribution converges in probability to $\tprate_1 \tprate_2$.  % = u - \Card{\Set{A}} - \Card{\Set{B}} + \Card{\Set{A} \cap \Set{B}}$.

Generally, the number of negatives $\n$ or positives $\p$ is not known, and so this serves a more analytic function, i.e., given around $\n$ negatives, what true negative rate $\tnrate$ provides the desired level of confidence that the true negative rate will not realize a value less than some specified value?




\begin{theorem}
	\label{thm:union_fn_rate}
	The union of $\ASet{A}[\tnrate_1][\fnrate_1]$ and $\ASet{B}[\tnrate_2][\fnrate_2]$ is a random approximate set with an expected false negative rate
	\begin{equation}
	\label{eq:union_fn_rate}
	\fnrate =
	\alpha_1 \fnrate_1 \tnrate_2 + \alpha_2 \tnrate_1\fnrate_2\\
	+ (1 - \alpha_1 - \alpha_2) \fnrate_1 \fnrate_2\,,
	\end{equation}
	where
	\begin{equation}
	\begin{split}
	\label{eq:union_fn_rate_alpha}
	0 &\leq \alpha_1 = \frac{\Card{\SetDiff[\Set{A}][\Set{B}]}}{\Card{\SetUnion[\Set{A}][\Set{B}]}}\,,\\
	0 &\leq \alpha_2 = \frac{\Card{\SetDiff[\Set{B}][\Set{A}]}}{\Card{\SetUnion[\Set{A}][\Set{B}]}}\,,\\
	\alpha_1 + \alpha_2 &\leq 1\,.
	\end{split}
	\end{equation}
\end{theorem}
See \cref{sec:union_fn_rate} for a proof of \cref{thm:union_fn_rate}.

The complement of an approximate set is given by the following theorem.
\begin{theorem}
	\label{thm:asetcompl}
	The \emph{complement} of a random approximate set with a false positive rate $\fprate$ and false negative rate $\fnrate$ is an approximate set with a false positive rate $\fnrate$ and a false negative rate $\fprate$.
\end{theorem}
\begin{proof}
	The false positives in an approximate set are false negatives in its complement;
	likewise, the false negatives in an approximate set are the false positives in its complement set.
\end{proof}

%Since many operations in $\PS{\Set{U}} \times \PS{\Set{U}} \mapsto \PS{\Set{U}}$ may be defined as a composition of unions and complements, 
%\cref{thm:union_fp,thm:union_fn_rate,thm:asetcompl} may be used to derive many other random approximate sets, such as intersections and set-difference.

\begin{remark}
	Consider a sequence $\PASet{A}[\fprate_1],\ldots,\PASet{A}[\fprate_n]$.
	Any \emph{subsequence} contains strictly less information about $\Set{A}$.
	That is, positive approximate sets are strictly \emph{additive}, and as $n \to \infty$, $\cap_{i=1}^{n} \PASet{A}[\fprate_i]$ converges almost surely to $\Set{A}$.\footnote{Likewise, for negative approximate sets, as $n \to \infty$, $\cup_{i=1}^{n} 
		\NASet{A}[\fnrate_i]$ converges almost surely to $\Set{A}$.}
\end{remark}

\end{document}