\documentclass[ ../main.tex]{subfiles}
\providecommand{\mainx}{..}
\begin{document}
\section{Distribution of binary classification measures}
\label{sec:perf}
%\label{sec:func_rand_asets}
Suppose we have some other function $\Fun{g} \colon \PS{\Set{X}} \mapsto \Set{Y}$ that is not a \emph{constant}, then the composition $\Fun{g} \circ \APFun{id}[\fprate][\tprate]$ is some probability distribution over the codomain $\Set{Y}$.
That is, $\left(\Fun{g} \circ \APFun{id}[\fprate][\tprate]\right)(\Set{A})$ is a \emph{random variable}.
\begin{example}
%Suppose we are interested in subsets of $\{0,1\}$ given by $\PS{\{0,1\}} = \left\{\EmptySet,\{0\},\{1\},\{0,1\}\right\}$.
%The random approximate set $\ASetStyle{\{0\}}[0.25][0.75]$ has a probability mass function $\Fun{p}[\ASetStyle{\{0\}}] \colon \PS{\{0,1\}} \mapsto [0,1]$ %given by
%\begin{equation}
%\Fun{p}[\ASetStyle{\{0\}}](\Set{B}) =
%\begin{cases}
%\frac{3}{16} & \Set{B} = \EmptySet\,,\\
%\frac{9}{16} & \Set{B} = \{0\}\,,\\
%\frac{1}{16} & \Set{B} = \{1\}\,,\\
%\frac{3}{16} & \Set{B} = \{0,1\}\,.
%\end{cases}
%\end{equation}
Let $\Fun{f} \colon \PS{\{0,1\}} \mapsto \{0,1\}$ be defined as
\begin{equation}
\Fun{f}(\Set{A}) \coloneqq
\begin{cases}
	1 & \Set{A} \in \left\{\{1\},\{0,1\}\right\}\,,\\
	0 & \text{otherwise.}
\end{cases}
\end{equation}
The composition $\Fun{f} \circ \APFun{id}[\mathsmaller{.25}][\mathsmaller{.75}]$ generates Bernoulli distribution random variables, e.g., $\left(\Fun{f} \circ \APFun{id}[\mathsmaller{.25}][\mathsmaller{.75}]\right)(\{0\}) \sim \berdist(0.25)$.
%As a function of a random variable, $\Fun{f}(\ASetStyle{\{0\}})$ is a random variable with a support $\{0,1\}$ given by the codomain of $\Fun{f}$.
%To determine the distribution over $\{0,1\}$, observe that on inputs $\{1\}$ and $\{0,1\}$, $\Fun{f}$ maps to $1$.
%Otherwise, $\Fun{f}$ maps to $0$.

%The random approximate set $\ASetStyle{\{0\}}$ maps to $\{1\}$ or $\{0,1\}$ with probability $0.25$.
%Therefore, $\Fun{f}(\ASetStyle{\{0\}})$ maps to $1$ with probability $0.25$ and maps to $1$ with probability $0.75$.
%Since $\Fun{f}$ is a Boolean function and each application of $\Fun{f}$ to $\ASetStyle{\{0\}}$ is independent and identically distributed, %$\Fun{f}(\ASetStyle{\{0\}})$ must be \emph{bernoulli distributed}, , i.e., $\Fun{f}(\ASetStyle{\{0\}}) \sim \berdist\left(0.25\right)$.
\end{example}

We consider several classes of functions and the distributions induced by replacing the inputs with random approximate sets, e.g., operators like set-union or binary performance measures like positive predictive value.
In \cref{sec:bool_search}, we consider a more sophisticated example in Boolean search where queries map to \emph{random approximate result sets}.

%Most set-theoretic operations, such as the subset or power set, may be defined with respect to the \emph{member-of} relation. Thus, the \emph{approximate} member-of relation induces other kinds of approximate relations.

In the approximate set model, the distribution of random variables like the false positive, false negatives, true positives, and true negative rates are given respectively by parameters $\fprate$, $\fnrate$, $\tprate$, and $\tnrate$.
These parameters belong to a more general class of \emph{binary performance measures}.

The above parameters are statements about the distribution of random approximate sets given corresponding objective sets of interest, e.g.,
\begin{equation}
\Prob{\SetIndicator{\ASet{A}[\fprate][\tprate]}(x) \Given 
	\SetIndicator{\Set{A}}(x)} = \tprate\,.
\end{equation}
%\begin{equation}
%    \Prob{x \in \ASet{S}(\fprate,\tprate) \Given x \in \Set{S}} = \tprate\,.
%\end{equation}

The accuracy of \emph{predictions} about objective sets given a corresponding approximate set is usually the more relevant performance measure.
The \emph{positive predictive value} is given by the following definition.
\begin{definition}
	The \emph{positive predictive value} is a performance measure defined as 
	\begin{equation}
	\ppv = \frac{t_p}{t_p + f_p}
	\end{equation}
	where $t_p$ is the number of \emph{true positives} and $f_p$ is the number 
	of \emph{false positives}.
\end{definition}

The positive predictive value of random approximate sets is a random variable given by the following theorem.
\begin{theorem}
	\label{thm:approx_expected_precision}
	Given $\n$ negatives, $\p$ positives, and a random approximate set with false positive and true positive rates $\fprate$ and $\tprate$ respectively, the \emph{positive predictive value} is a random variable
	\begin{equation}
		\PPV = \frac{\TP_{\p}}{\TP_{\p} + \FP_{\n}}
	\end{equation}
	with an \emph{expectation} given \emph{approximately} by
	\begin{equation}
		\label{eq:approx_expected_precision}
		\ppv(\tprate, \fprate, \p, \n) \approx 
		\frac{\overline{t}_\p}{\overline{t}_\p + \overline{f}_\p} +
		\frac{\overline{t}_\p \sigma_{\!f_\p}^2 - \overline{f}_\p 
		\sigma_{\!t_\p}^2}{\left(\overline{t}_\p + \overline{f}_\p\right)^3}\,,
	\end{equation}
	where $\overline{t}_\p = \p \tprate$ is the expected \emph{true positive} frequency, $\overline{f}_\p =  \n \fprate$ is the expected \emph{false positive} frequency, $\sigma_{\!t_\p}^2 = (1-\tprate) \overline{t}_\p$ is the variance of the \emph{true positive} frequency, and $\sigma_{\!f_\p}^2 = (1-\fprate) \overline{t}_\p$ is the variance of \emph{false positive} frequency.
\end{theorem}
See \cref{sec:proof_approx_expected_precision} for a proof of 
\cref{thm:approx_expected_precision}.

\begin{figure}
	\def\svgwidth{\columnwidth}
	%\def\svgscale{0.75}
	\centering
	\captionsetup{justification=centering}
	\caption
	{
		Relative frequency of positive predicitive values for several different parameterizations of the false positive and true positive rates given $\n = 900$ negatives and $\p=100$ positives.
	}    
	\input{img/out.pdf_tex}
	\label{fig:prec_vs_fprate_and_fnrate}
\end{figure}

We make the following observations about \cref{eq:approx_expected_precision}:
\begin{enumerate}
	\item For sufficiently large approximate sets, $\ppv \approx 
	\overline{t}_\p / (\overline{t}_\p + \overline{f}_\p)$.
	\item If $\fprate \neq 0$, as $\n \to \infty$, $\ppv \to 0$.
	\item As $\fprate \to 0$, $\ppv \to 1$.
\end{enumerate}

\emph{Accuracy} is given by the following definition.
\begin{definition}
	The \emph{accuracy} is the proportion of true results (both \emph{true 
	positives} and \emph{true negatives}) in the universe of positives and 
	negatives, $(t_p + t_n)/(\p + \n)$, where $t_p$, $t_n$, $\p$, and $\n$ are 
	respectively the number of \emph{true positives}, \emph{true negatives}, 
	\emph{positives}, and \emph{negatives}.
\end{definition}

The \emph{expected} accuracy is given by the following theorem.
\begin{theorem}
	\label{thm:approx_expected_accuracy}
	Given $\p$ positives and $\n$ negatives, a random approximate set with an 
	\emph{expected} false positive rate $\fprate$ and an \emph{expected} true 
	positive rate $\tprate$ is a random variable given by
	\begin{equation}
		\ACC_{\p + \n} = \lambda \TPR_\p + (1 - \lambda) \TNR_\n\,.
	\end{equation}	
	has an \emph{expected} accuracy
	\begin{equation}
	\label{eq:approx_expected_accuracy}
	\acc(\tprate,\fprate,\n,\p) = \lambda \tprate + (1-\lambda) \tnrate
	\end{equation}
	with a variance
	\begin{equation}
	\frac{\lambda \fnrate \tprate + (1 - \lambda)\fprate 
		\tnrate}{\p+\n}\,,   
	\end{equation}
	where $\lambda = \p / (\p + \n)$.
\end{theorem}
\begin{proof}
	Suppose there the $u$ elements in the universe can be partitioned into $\p$ 
	positives and $\n$ negatives. An approximate set $\ASet{S}$ with a false 
	positive rate $\fprate$ and false negative rate $\fnrate$ has an uncertain 
	accuracy
	\begin{equation}
	\ACC_{\p + \n} = \frac{\TP_\p + \TN_\n}{\p + \n}\,.
	\end{equation}
	The expected accuracy is given by the expectation
	\begin{align}
	\Expect{\ACC_{\p + \n}}
	&= \Expect{\frac{\TP_\p + \TN_\n}{\p + \n}}\\
	&= \frac{\p (1 - \fnrate) + \n (1 - \fprate)}{\p + \n}\,.
	\end{align}
	Noting that $\n/(\p+\n) = 1 - \p/(\p+\n)$ and letting $\lambda = \p/(\p + 
	\n)$,
	\begin{equation}
	\Expect{\ACC_{\p + \n}} = \lambda (1 - \fnrate) + (1 - \lambda) (1 - 
	\fprate)\,.
	\end{equation}
	The variance
	\begin{align}
	\Var{\ACC_{\p + \n}}
	&= \Var{\frac{\TP_\p}{\p + \n}} + \Var{\frac{\TN_\n}{\p + \n}}\\
	&= \frac{1}{(\p + \n)^2}\Var{\TP_\p} + \frac{1}{(\p + \n)^2}\Var{\TN_\n}\\
	&= \frac{\p \fnrate (1 - \fnrate)}{(\p + \n)^2} + \frac{\n \fprate (1 - 
	\fprate)}{(\p + \n)^2}\\
	&= \frac{\lambda \fnrate \tprate + (1 - \lambda)\fprate 
		\tnrate}{\p+\n}\,.
	\end{align}
\end{proof}

\emph{Negative predictive value} is given by the following definition.
\begin{definition}
	\begin{equation}
	\npv = \frac{t_n}{t_n + f_n}
	\end{equation}
	where $t_n$ and $f_n$ are respectively he number of \emph{true negatives} 
	and \emph{false negatives} 
\end{definition}
The expected negative predictive value is given by the following theorem.
\begin{theorem}
	\label{thm:npv_approx}
	Given $\p$ positives, $\n$ negatives, and a random approximate set with false positive and true positive rates $\fprate$ and $\tprate$ respectively, the negative predictive value is a \emph{random variable}
	\begin{equation}
		\NPV = \frac{\TN_{\n}}{\TN_{\n} + \FN_{\p}}
	\end{equation}
	with an \emph{expectation} given approximately by
	\begin{equation}
	\label{eq:npv_approx}
	\npv(\tprate, \fprate , \p, \n) \approx 
	\frac{\overline{t}_n}{\overline{t}_n + \overline{f}_{\!n}} +
	\frac{\overline{t}_n \sigma_{\!f_n}^2 - \overline{f}_{\!n} 
	\sigma_{\!t_n}^2}{\left(\overline{t}_n + \overline{f}_{\!n}\right)^3}\,,
	\end{equation}
	where $\overline{t}_\n = \n(1-\fprate)$ is the expected \emph{true negative} frequency, $\overline{f}_{\!\n} =  \p(1-\tprate)$ is the expected \emph{false negative} frequency, $\sigma_{\!t_\n}^2 = \fprate \overline{t}_\n$ is the variance of the \emph{true negative} frequency, and $\sigma_{\!f_\n}^2 = \tprate\overline{f}_{\!\n}$ is the variance of the \emph{false negative} frequency.
\end{theorem}
The proof for \cref{thm:npv_approx} follows the same pattern as the proof for \cref{thm:approx_expected_precision}.

Youden's $J$ statistic is a measure of the performance of a binary test, defined as
\begin{equation}
    J = \frac{t_p}{t_p + f_n} + \frac{t_n}{t_n + f_p} - 1\,,
\end{equation}
with a range $[0,1]$.
In the case of the random approximate set model, $J$ is a random variable
\begin{equation}
	\RV{J} = \TPR_\p - \FPR_\n\,,
\end{equation}
which has an \emph{expectation}
\begin{equation}
	\Expect{\RV{J}} = \tprate - \fprate\,.
\end{equation}

\begin{table}
	\centering
	\caption{Various \emph{expected} performance measures.}
	\label{tbl:perf_sum}    
	\begin{tabular}{@{} l l l @{}}
		\toprule
		\textbf{measure} & \textbf{parameter} & \textbf{expected value}\\
		\midrule
		true positive rate & $\tpr(\tprate)$ & $\tprate$\\
		false positive rate & $\fpr(\fprate)$ & $\fprate$\\
		false negative rate & $\fnr(\tprate)$ & $1-\tprate$\\
		true negative rate & $\tnr(\fprate)$ & $1-\fprate$\\
		accuracy & $\acc$ & \cref{eq:approx_expected_accuracy}\\        
		positive predictive value & $\ppv$ & \cref{eq:approx_expected_precision}\\
		negative predictive value & $\npv$ & \cref{eq:npv_approx}\\        
		false discovery rate & $\fdr$ & $1-\ppv$\\
		false omission rate & $\forFn$ & $1-\npv$\\
		\bottomrule
	\end{tabular}
\end{table}
\Cref{tbl:perf_sum} may be used to reparameterize an approximate set.
\begin{example}
	Suppose we seek a \emph{positive approximate set} with an expected accuracy 
	$\gamma$. By \cref{tbl:perf_sum},
	\begin{equation}
	\gamma = \acc(\fprate,\fnrate=0,\lambda) = 1 - \fprate(1 - \lambda)\,.
	\end{equation}
	% Y = 1 - e(1- lam)
	% (1-Y)/(1-lam) = e
	Solving for $\fprate$ in terms of $\gamma$ yields the result
	\begin{equation}
	\fprate(\gamma, \lambda) = \frac{1 - \gamma}{1 - 
		\lambda}\
	\end{equation}
	subject to $0 \leq \lambda \leq \gamma \leq 1$ and $\lambda < 1$.
	Under this parameterization of the 	positive approximate set, $\lambda$ 
	must be known (or estimated).
	Note that if $\lambda = 1$ then $\fprate(\gamma, \lambda=1)$ is undefined as expected, but as $\lambda$ goes to $1$, $\fprate(\,\cdot\,; \lambda)$ goes to $1$ and $\gamma$ goes to $1$, which logically follows since if there are no negatives, there can be no false positives.
\end{example}

\end{document}