\documentclass[ ../main.tex]{subfiles}
\providecommand{\mainx}{..}
\begin{document}
\section{Introduction}
A fuzzy approximate set is a fuzzy set that approximates another fuzzy set of actual interest. It is \emph{approximate} because with respect to the actual fuzzy set, there are two types of errors, \emph{false positives} and \emph{false negatives}.

In \Cref{dummyref}, we precisely define the fuzzy approximate set. In \Cref{dummyref} and \Cref{dummyref}, we respectively explore properties of \emph{finite} approximate sets and \emph{countably infinite} approximate sets, such as their \emph{precision} and \emph{recall}. In \Cref{dummyref}, we consider set-theoretic unions, intersections, and complements of approximate sets, which generate approximate sets with well-defined false positive and false negative rates. Finally, in \Cref{dummyref}, we explore approximate Boolean search based on positive approximate sets. To prove various properties of approximate Boolean search, such as precision, we only need to prove that the \emph{result sets} are positive approximate sets of the \emph{true} results and everything else follows as a result.
\end{document}