\documentclass[ ../main.tex]{subfiles}
\providecommand{\mainx}{..}
\begin{document}
\section{Fuzzy set-theoretic model}
In the fuzzy set-theoretic model, any fuzzy set may be defined as a composition of other fuzzy sets. Any composition, in turn, may be reduced to fuzzy interesections and complements (or fuzzy unions and complements).

The fuzzy set-theoretic operators operate on degree-of-membership values. Fuzzy operator equivalents to Boolean $\notfn$, $\orfn$, and $\andfn$ are given by
\begin{align}
    \notfn(x) &= 1 - x\,,\\
    \orfn(x, y) &= \max(x, y)\,,\\
    \andfn(x, y) &= \min(x, y)\,,
\end{align}
where $x$ and $y$ are degree-of-membership values.


For instance, fuzzy set-difference, $\Sa_1 \setminus \Sa_2$, the approximate fuzzy set consisting of elements in $\Sa_1$ but not in $\Sa_2$, is given by
\begin{equation}
    
\end{equation}


A hedge function transforms degree-of-membership values as demonstrated by the following example.
\begin{example}
It may be true that an element is a member of a set $\St$ but it may not be \emph{very} true. If we are interested in sets in which it is \emph{very} true that an element is a member, then the hedge function given by
\begin{equation}
    \very(x) = x^2
\end{equation}
makes the appropriate transformation since $\True\left(\very(x)\right)$ is \emph{false} for larger values of $x$ than $\true(x)$.
\end{example}

A Boolean membership test is eventually needed, in which case defuzzification transforms degree-of-membership values into crisp $\true$ or $\false$ values. For instance,
\begin{equation}
    \true(x) = x \geq K\,,
\end{equation}
where $K \in [0,1]$, transforms degree-of-membership values larger than $K$ to $\true$ and otherwise $\false$. At this point, standard set-theoretic queries as given by BNF~\ref{bnf:set_theoretic} may be used.


Fuzzy set-theoretic queries operate on degree-of-membership values given by $\operatorname{u}$. Fuzzy operator equivalents to Boolean $\notfn$, $\orfn$, and $\andfn$ are given by
\begin{align}
    \notfn(x) &= 1 - x\,,\\
    \orfn(x, y) &= \max(x, y)\,,\\
    \andfn(x, y) &= \min(x, y)\,,
\end{align}
where $x$ and $y$ are degree-of-membership values.

A Boolean membership test is eventually needed, in which case defuzzification transforms degree-of-membership values into crisp $\true$ or $\false$ values. For instance,
\begin{equation}
    \TrueFn(x) = x \geq K\,,
\end{equation}
where $K \in [0,1]$, transforms degree-of-membership values larger than $K$ to $\True$ and otherwise $\False$. At this point, standard set-theoretic queries as given by BNF~\ref{bnf:set_theoretic} may be used.



\end{document}