\documentclass[ ../main.tex]{subfiles}
\providecommand{\mainx}{..}
\begin{document}
\section{Random approximate maps}
The set of functions of type $\Set{X} \mapsto \Set{Y}$ has cardinality $\Card{\Set{Y}}^{\Card{\Set{X}}}$.
We may also denote functions of this type by $\Set{Y}^{\Set{X}}$.
Given a function $\Fun{f} \colon \Set{X} \mapsto \Set{Y}$, a randomly chosen function $\Fun{g} \colon \Set{X} \mapsto \Set{Y}$ equals $\Fun{f}$ when applied to $x \in \Set{X}$ with probability $\frac{1}{\Card{\Set{Y}}}$ and therefore $\Fun{g} = \Fun{f}$ with probability $\Card{\Set{Y}}^{-\Card{\Set{X}}}$.
%More generally, the number of elements in the domain $\Set{X}$ where $\Fun{g}$ and $\Fun{f}$ match is a binomially distributed random variable.

We generalize this result and consider \emph{random approximate maps}.

\begin{definition}
The Iverson bracket, denoted by $\llbracket \, \rrbracket \colon \{\True,\False\} \mapsto \BitSet$, maps a predicate to $1$ if true and otherwise $0$.
\end{definition}
For example, $\llbracket a<b \rrbracket = 1$ if $a<b$ is \True.

Let $\RV{Y}_{x} \coloneqq \llbracket\Fun{f}(x)=\APFun{f}(x)\rrbracket$ be a Boolean random variable.
The probability that for all $x$ in $\Set{X}$, $\RV{Y}_{x} = 1$ is given by
\begin{equation}
	\Prob{\cap_{x \in \Set{X}} \RV{Y}_{x}=1}\,.
\end{equation}
If we assume independence,
\begin{equation}
	\Prob{\cap_{x \in \Set{X}} \RV{Y}_{x}=1} = \prod_{x \in \Set{X}} \Prob{\RV{Y}_{x}=1}\,.
\end{equation}
Since $\RV{Y}_{x}$ for each $x \in \Set{X}$ denote independent Boolean random variables, they are \emph{Bernoulli} distributed.

\begin{definition}
TODO: talk about independence.
The \emph{order} of a random approximate map corresponds to the minimum number of partitions in which every element of a given partition is identically and independently Bernoulli distributed random variables.
We model $\Fun{f} \colon \Set{X} \mapsto \Maybe{\Set{Y}}$ by a \emph{random} approximate function denoted by $\Fun{\AT{f}}$.
That is, $\Fun{\AT{f}}$ is a random function.
\end{definition}

We denote a $k$-th order random approximate map of $\Fun{f}$ by $\Fun{f}[k][\pm]$ or, if we know and wish to emphasize the partitions, $\Fun{f}[\Set{X}[1],\ldots,\Set{X}[k]][\pm]$ or the Bernoulli probabilities corresponding to each partition, $\Fun{f}[p_1,\ldots,\p_k][\pm]$.



Since each partition consists of some number of Bernoulli distributed random variables, that means each partition has a \emph{binomially distributed} number of $1$'s.



%\begin{remark}
%	NOTE: Is a first-order model plausible? Suppose we have variable length codes? If we condition on $\PlainSet{N}$ for instance, then $\Fun{\ell}_{\RV{X} %\Given \PlainSet{N}}(\Fun{f}, \APFun{f}[\fprate][\fnrate]) = \fprate$, but $\Prob{\APFun{f}[\fprate][\fnrate](\RV{X}) \neq \Nothing \Given \RV{X} \in %\PlainSet{N}} = \fprate$ does seem to be a Bernoulli trial $\berdist(\fprate)$.
%\end{remark}



In the random approximate map model, we assume that $\RV{X}$ is uniformly distributed.


\subsection{Approximate set-indicator function}
We may view a set $\Set{A}$ with respect to its \emph{indicator} function $\SetIndicator{\Set{A}} \colon \Set{X} \mapsto \BitSet$.

The set indicator function partitions $\Set{X}$ into two parts, $\PlainSet{P} \coloneqq \SetBuilder{x \in \Set{X}}{\SetIndicator{\Set{A}}(x) = 1}$ and $\PlainSet{N} \coloneqq \SetBuilder{x \in \Set{X}}{\SetIndicator{\Set{A}}(x) = 0}$.



%One in particular is defined as $\Fun{d}(a,a) = 1$ for any $a$ and otherwise $0$, in which case $\Fun{\ell}_{\RV{X} \Given \FancySet{A}}(\Fun{g} \Given \Fun{f})$ is the probability that $\Fun{f}(\RV{X}) \neq \Fun{g}(\RV{X})$ given that $\RV{X} \in \FancySet{A}$.
%Conditioning on different subsets of $\Set{X}$ yields different \emph{binary performance measures}.

\begin{definition}
\label{def:fpr}
The \emph{false positive rate}, denoted by $\hat{\fprate}$, is defined as the expectation $\Fun{\ell}_{\RV{X} \Given \PlainSet{N}}(\Fun{g} \Given \Fun{f})$ and the \emph{true negative rate}, denoted by $\hat{\tnrate}$, is defined as the probability $1-\hat{\fprate}$.
\end{definition}

\begin{definition}
	\label{def:fnr}
	The \emph{false negative rate}, denoted by $\hat{\fnrate}$, is defined as the expectation $\Fun{\ell}_{\RV{X} \Given \PlainSet{P}}(\Fun{g} \Given \Fun{f})$ and the \emph{true positive rate}, denoted by $\hat{\tprate}$, is defined as the probability $1-\hat{\tprate}$.
\end{definition}

\begin{remark}
	If $\PlainSet{P}$ is a priori known, we can ignore elements not in the domain of definition of $\Fun{f}$ and thus discard $\hat{\fprate}$.
\end{remark}


By \cref{eq:disj_loss,def:fpr,def:fnr}, the expected loss over the entire domain is given by
\begin{equation}
\Fun{\ell}_{\RV{X}}(\Fun{g} \Given \Fun{f}) = \Fun{\ell}_{\RV{X} \Given \PlainSet{N}}(\Fun{g} \Given \Fun{f}) \Fun{p}_{\RV{X}}(\PlainSet{N}) + \Fun{\ell}_{\RV{X} \Given \PlainSet{P}}(\Fun{g} \Given \Fun{f}) \Fun{p}_{\RV{X}}(\PlainSet{P}) = \hat{\fprate} (1-\Fun{p}_{\RV{X}}(\PlainSet{P})) + \hat{\fnrate} \Fun{p}_{\RV{X}}(\PlainSet{P})\,.
\end{equation}




The false negative rate is given by $\fnrate = \Fun{\ell}_{\RV{X} \Given \PlainSet{P}}(\Fun{f}, \Fun{\AT{f}})$.

The false positive rate is given by $\fprate = \Fun{\ell}_{\RV{X} \Given \PlainSet{N}}(\Fun{f}, \Fun{\AT{f}})$.

Each element in $\PlainSet{N}$ is a false positive with probability $\fprate$.
That is, it is a Bernoulli trial with $p = \fprate$.
If a false positive occurs, then it randomly chooses a \emph{representation} in the prefix-free codes for $\Set{Y}$ or an undefined code, which we map to \Nothing.

If the values in $\Set{Y}$ are uniquely represented, then it randomly (uniformly) chooses a value in $\Set{Y}$.

It may hash to an undefined code.
Two possibilities to consider:
\begin{enumerate}
	\item The hash to undefined code that is not a prefix of any other code.
	\item If the codes are variable-length, it may hash to a prefix of one or more codes.
\end{enumerate}

If the values in $\Set{Y}$ are uniquely represented, then it randomly (uniformly) chooses a value in $\Set{Y}$.








\subsection{Probability model}

TODO 3:


\COMMENT{
\begin{example}
	Suppose we have two functions, $\Fun{f} \colon \{a,b,c\} \mapsto \{0,1,2\}$ and $\Fun{g} \colon \{0,1,2\} \mapsto \{\triangle,\mathlg{\diamond},\square\}$ defined respectively as
	\begin{multicols}{2}
	\noindent
	\begin{equation}
		\Fun{f}(x) \coloneqq
		\begin{cases}
		1	&	x = a\,,\\
		2	&	x = b\,,\\
		0	&	x = c\,.		
		\end{cases}
	\end{equation}
	\begin{equation}
		\Fun{g}(y) \coloneqq
		\begin{cases}
		\triangle			&	y = 0\,,\\
		\mathlg{\diamond}	&	y = 1\,,\\
		\square				&	y = 2\,.
		\end{cases}
	\end{equation}
	\end{multicols}
	
	We approximate $\Fun{f}$ with $\Fun{f}[\PAT{\{a,b\}}] \colon \{a,b,c\} \mapsto \{0,1,2\}$ and $\Fun{g}$ with $\Fun{g}[\PAT{\{0\}}] \colon \{0,1,2\} \mapsto \{\triangle,\mathlg{\diamond},\square\}$ respectively with the probability mass functions defined as
	\begin{multicols}{2}
		\noindent
		\begin{equation}
			\Fun{f}[\PAT{\{a,b\}}](x) =
			\begin{cases}
			0													&	x = a\,,\\
			2													&	x = b\,,\\
			\RV{Y} \sim \Fun{p}[\RV{Y} \Given x \notin \{a,b\}]	&	x = c\,.
			\end{cases}
		\end{equation}
		\begin{equation}
			\Fun{p}(x \Given x \in \{a,b\}) =
			\begin{cases}
			0	&	7/8\,,\\
			1	&	1/16\,,\\
			2	&	1/16
			\end{cases}
		\end{equation}
	\end{multicols}

	\begin{multicols}{2}
		\noindent
		\begin{equation}
		\Fun{p}(x \Given x \in \{a,b\}) =
		\begin{cases}
		0	&	7/8\,,\\
		1	&	1/16\,,\\
		2	&	1/16
		\end{cases}
		\end{equation}
		\begin{equation}
		\Fun{p}(x \Given x \in \{0\}) =
		\begin{cases}
		\triangle			&	1/2\,,\\
		\mathlg{\diamond}	&	1/3\,,\\
		\square				&	1/6\,.
		\end{cases}
		\end{equation}
	\end{multicols}

	
	The composition $\Fun{f} \circ \Fun{g} \colon \{a,b,c\} \mapsto \{\triangle,\mathlg{\diamond},\square\}$ and its probability mass are respectively given by
	\begin{equation}
	(\Fun{f} \circ \Fun{g})(x) \coloneqq
	\begin{cases}
	\mathlg{\diamond}	&	x = a\,,\\
	\square				&	x = b\,,\\
	\triangle			&	x = c\,.
	\end{cases}
	\end{equation}
	
\end{example}
}









General first-order random approximate map model is given by the following definition.
\begin{definition}
A function $\Fun{f}[\PAT{\Set{A}}] \colon \Set{X} \mapsto \Set{Y}$, $\Set{A} \subseteq \Set{X}$, is a \emph{first-order} random approximate map of $\Fun{f} \colon \Set{X} \mapsto \Set{Y}$ if the following conditions are satisfied:
\begin{equation}
\Fun{f}[\AT{\Set{A}}](x) \coloneqq
	\begin{cases}
		\RV{Y^{+}}		&	x \notin \Set{A}\,,\\
		\RV{Y_{-}}		&	x \in \Set{A}\,.
	\end{cases}
\end{equation}
where $\RV{Y^{+}}$ is a random variable with a support that is a subset of $\Set{Y}$ (including being equal to $\Set{Y}$).
\end{definition}

If we have a first-order rate-distorted map, then $\RV{Y^{+}}$ has a probability mass function that is given by
\begin{equation}
	\Fun{p}_{\RV{Y^{+}}}(y) = \sum_{b \in \Encode_{\Set{Y}}(y)} 2^{-\BL(b)}\,,
\end{equation}
where $\Encode_{\Set{Y}}$ is a prefix-free encoder for values in $\Set{Y}$.

An optimal encoder may be found by computing the frequency of each element in $\Set{Y}$ when we apply $\Fun{f}$ to every element in $\Set{A}$ or $\Set{X}$\footnote{The latter may be chosen if we care about the approximation error over $\SetDiff[\Set{X}][\Set{A}]$.} and generating a Huffman code for it.

However, since encoding the Huffman code may require a relatively large amount of memory, especially if the image of $\Fun{f}$ is large, this approach may not be practical.
Alternatively, there are many well-known algorithms to generate prefix codes for specific distributions, or families of distributions.
Universal codes may be appropriate when the image of $\Fun{f}$ is very large.
If all else fails, there is always the uniform distribution.

\begin{example}
The first-order positive random approximate set $\PASet{A}[\fprate]$ is a first-order random approximate map of $\SetIndicator{\Set{A}} \colon \Set{X} \mapsto \BitSet$ where, conditioning on $\Set{A}$, $\sum_{b \in \Encode_{\BitSet}(1)} 2^{-\BL(b)} = \fprate$ and $\sum_{b \in \Encode_{\BitSet}(0)} 2^{-\BL(b)} = 1-\fprate$.
\end{example}



If there are constraints, such as the space of bijective functions $\Set{X} \mapsto \Set{X}$, then there are $\Card{\Set{X}}!$ such functions and the probability that, given a bijective function $\Fun{f}$, a randomly chosen bijective function $\Fun{g}$ is equal to $\Fun{f}$ with probability $\frac{1}{\Card{\Set{X}}!}$.




TODO -1:

The distance function $\Fun{d}(a,a) = 0$ and otherwise $1$ combined with $\Fun{\ell}$ counts the proportion of values mapped correctly.

TODO 0:

The \emph{zero-th} order approximation? Maybe we say that its uniformly distributed, i.e., just a random variable with a sample space $\Set{Y}$.

What is first order then? Partition into $k$ subsets, and elements in partition $j$ are identically distributed.

What is second order? Partition into $k$ subsets, and when we condition on a subset ...

TODO .5: 
Is 1st order actually 4th order model? Partions $\Set{X}$ into $\Set{A}$ and $\SetComplement[\Set{A}]$. Then, it partitions $\SetComplement[\Set{A}]$ into $\{y^*\}$ and $\SetComplement{\Set{A}} - \{y^*\}$ and it partitions $\Set{A}$ given $x \in \Set{A}$ into $\{\Fun{f}(x)\}$ and $\Set{Y} - \{\Fun{f}(x)\}$. That makes 4 partitions, each element in which has a different conditional probability.

We shouldn't overcomplicate though. Maybe just say these are random approximate maps, sometimes difficult to quantify distributions, but in some cases we can provide a well-understood model, e.g., the set-indicator function has a first-order model that's straight-forward. Even higher-order models are somewhat understandable. Boolean algebra types in general, including $\{0,1\}$, all have this easily understood first-order model?



TODO 1:

Revised first-order model: $\Fun{f}_{\AT{A}} \colon \Set{X} \mapsto \Set{Y}$ has the property that if $x \in \Set{A}$ then $\RV{Y} = \Fun{f}_{\AT{\Set{A}}}(x)$ realizes $\Fun{f}(x)$ with probability $1-r$ and is uniformly distributed over sample space $\Set{Y} - \{\Fun{f}(x)\}$ otherwise, i.e., realizes any value in $\Set{Y} - \{\Fun{f}(x)\}$ with probability $r / (\Card{\Set{Y}}-1)$.

For each $x \notin \Set{A}$, $\RV{Y} = \Fun{f}_{\AT{\Set{A}}}(x)$ is independently and identically distributed where it realizes some default value $y^*$ with probability $1-\fprate$ and is uniformly distributed over $\Set{Y} - \{y^*\}$ otherwise.

TODO 1.5:

Don't necessarily mention rate distortion except in passing, and then talk about, say, SHM as generating errors due to rate distortion.


TODO 2:

Given a function $\Fun{f} \colon X \mapsto Y$, $\Fun{f}[\AT{Q}[\fprate][\fnrate]][q] \colon X \mapsto Y_q$ approximately models $\Fun{f} \restriction_{Q} \colon X \mapsto Y$ where elements not in $Q \subseteq X$ are mapped by $\Fun{f} \restriction_{\AT{Q}}$ to $Y$ with probability $\fprate$ and to some \emph{default} value $q$ with probability $1-\fprate$.

The \emph{first-order random approximate set}, $\ASet{A}[\fprate][\fnrate]$, may be modeled by the approximate map of the characteristic function, $\Fun{f}[\ASet{A}[\fprate][\fnrate]][0] \colon \Set{U} \mapsto \BitSet$.

If we wish to speak of the \emph{type}, we may move the notation to the arrow, i.e., $X \amapsto{\fprate}{\fnrate} Y$.

The default value may be disjoint from $Y$, e.g., the sum type $Y + q$.

If we do not know the rates, or they are unimportant, we may instead denote it by $\Fun{f}[\AT{Q}] \colon X \mapsto Y$.


$\Fun{f} \colon X \mapsto Y$, $\Fun{f}[\AT{Q}[\fprate][\fnrate]] \colon X \mapsto Y$

We say that is of type 
$\Fun{f} \colon X \mapsto Y$, $\Fun{f}[\AT{Q}[\fprate][\fnrate]] \colon X \mapsto Y$


Additionally, elements in $Q$ are not mapped by $\Fun{f}_{\restriction_{\AT{Q}}}$.



TODO: revisit false negative and false positive rate. False positive rate should probably just be about what the approximate map is conditioned on, i.e., if f : X -> Y and we build amap(f,A) then we have fA : A -> m(Y). If fA(x) != Nothing when x not in A, then that is a false positive. Simplify this result. Now we can focus more on false negatives. A false negative is when fA(x) != f(x) when x in A. In other words, we want to look at the restriction of the original function f to A -> m(Y) (partial) and compare it to fA.

Now, for the approx boolean, fA : bool -> bool (or m(bool)), we typically let A = {0,1}. So, in that case, there are no "negatives" and so no false positives, there is only a rate distortion on the positives. Only two positives, 0 and 1. If fA(false) maps to Nothing, that's a false negative... but shouldn't really happen in this case the function is so simple. Now we must also talk about the mabye monad though... because a approx(bool) is {0,1,nothing} with some prob. distribution over the elements, P[fA(true) = nothing] is false negative rate. P[fA(true) != nothing] is 1-fnr. Rate distortion on positives that don't map to Nothing are a different matter. Now we're talking about 


\end{document}