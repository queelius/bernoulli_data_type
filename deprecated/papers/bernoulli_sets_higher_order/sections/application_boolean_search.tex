\documentclass[ ../main.tex]{subfiles}
\providecommand{\mainx}{..}

\SetKwFunction{MakeIndex}{index}
\SetKwFunction{MakeApproxIndex}{index$^\sigma$}
\SetKwFunction{OrOp}{or}
\SetKwFunction{NotOp}{not}
\SetKwFunction{AndOp}{and}
%\SetKwFunction{EncryptedFind}{hidden\_find$^\sigma$}

\begin{document}
\section{Application: approximating Boolean search}
\label{sec:bool_search}
%TODO: let's do the subset relation instead! just to make it a little more sophisticated/interesting. refer to oblivious types also, i.e., a hidden query
%      is an oblivious type and the secure indexes are oblivious sets of the oblivious types. in this case, the oblivious types represent elements of the
%      powerset of keys up to k-graphs?     

An information retrieval process begins when a user submits a \emph{query} to an information system, where a query represents an \emph{information need}.
In response, the information system returns a set of relevant documents that satisfy the query.

Boolean search is an information retrieval model given by the following definition.
\begin{definition}
A document in the collection is either \emph{relevant} or \emph{non-relevant} to a Boolean query.
\end{definition}
We do not specify the structure of documents since we are only interested in being able to specify documents in a collection by some \emph{label}, e.g., a file name.
We specify the universal set of document labels by $\Set{D}$ and therefore a \emph{particular} collection of interest is a subset of $\Set{D}$.

We consider queries over the Boolean algebra $Q = \left(\PS{\Set{K}}, \land, \lor, \neg, \epsilon, \Set{K}\right)$, where $\Set{K}$ denotes a set of \emph{search keys}, e.g., units of information like English words.\footnote{This is isomorphic to the Boolean algebra $\left(\{0,1\}^k, \land,\lor,\neg,0^k,1^k\right)$ where $k=\Card{\Set{K}}$.}
Without loss of generality, we transform Boolean queries over $Q$ to the BNF
\begin{align*}
\langle\text{query} \rangle & \coloneqq
	\text{``} \langle \text{key} \rangle\text{''} \mid \neg \left(\langle \text{query} \rangle \right) \mid\\
	& \qquad \lor \left(\langle \text{query} \rangle\,,\langle \text{query} \rangle\right) \mid\\
	& \qquad \land \left(\langle \text{query} \rangle\,,\langle \text{query} \rangle\right)\\
\langle\text{key}\rangle & \coloneqq \text{a key in $\Set{K}$}\,.
\end{align*}

\emph{Search indexes} may be used to quickly compute whether a given document is relevant to a given query.
Since we are using a Boolean search query model $Q$, search indexes may be efficiently represented  by \emph{sets} over $\Set{K}$ in the Boolean algebra $S = (\PS{\Set{K}}, \SetIntersection, \SetUnion, \SetComplement, \EmptySet, \Set{K})$.
In particular, let $\MakeIndex \colon \Set{D} \mapsto \PS{\Set{K}}$ be a function that maps documents to search indexes with a definition given by
\begin{equation}
	\MakeIndex(d) \coloneqq \SetBuilder{k \in \Set{K}}{	\text{$k$ is relevant to $d$}}\,.
\end{equation}

The set of relevant documents to a query is denoted the query's \emph{result set}.
The result sets form the Boolean algebra $R = (\PS{\Set{D}}, \SetIntersection, \SetUnion, \SetComplement, \EmptySet, \Set{D})$.

A bijection from $Q$ to $S$ is given by $\land \mapsto \SetIntersection$, $\lor \mapsto \SetUnion$, $\neg \mapsto \SetComplement$, $\epsilon \mapsto \EmptySet$, and $\Set{K} \mapsto \Set{K}$.
Let $\Find \colon Q \times \PS{\Set{D}} \mapsto \PS{\Set{D}}$ be the function that maps queries in $Q$ to result sets in $R$ by using the collection of corresponding search indexes in $S$,
\begin{equation}
\Find(q,ds) \coloneqq
\begin{cases}
	\SetComplement\left(\Find{$t$,$ds$}\right) & \text{if $h = \neg$} \\
	\SetUnion
		\left(
			\Find{\Left{$t$},$ds$},
			\Find{\Right{$t$},$ds$}
		\right) & \text{if $h = \lor$} \\
	\SetIntersection
		\left(
			\Find{\Left{$t$},$ds$},
			\Find{\Right{$t$},$ds$}
		\right) & \text{if $h = \land$} \\
	\SetBuilder{d \in ds}{h \in \MakeIndex(d)} & \text{otherwise}\,,
\end{cases}	
\end{equation}
where $h = \Head(q)$, $t = \Tail(q)$, $\Head \colon Q \mapsto \SetUnion[\{\neg,\lor,\land\}][\Set{K}]$ maps any given query $q$ to the next \emph{Boolean} operation or \emph{key} in $q$, $\Tail \colon Q \mapsto Q$ maps any given query $q$ to nested sub-queries in $q$, e.g., $\Tail(\lor(q_1,q_2)) = ?$, \Left maps $\operatorname{f}(x,y)$ to $x$ and $\Right$ maps $\Fun{f}(x,y)$ to $y$.

%TODO: x xor y = (x or y) and not (x and y)
% blindly apply the transformation rules for AND, OR, and NOT, we get:
%     (x + y - xy)(1 - xy) = x + y - xy - x^2y - xy^2 +x^2y^2
% replace v^2 by v
%     x + y - xy - xy - xy + xy = x + y - 2xy
% other rules: x^2 = x, (x + x) = x .. so, hmm? seems like it should be x + y - xy?
%
% better approach: Karnaugh maps. use this algorithm and other properties (empty set + A = A)
% to simplfy any boolean expression as much as possible,
% then use some ordering relation to order the connectives and terminal sets so that any two equivalent
% set-theoretic expressions on the same sets reduce to the same form.



%
% what is the iterated function f applied repeated to approximate set A',B'? it converges
% to something... particular examples: f is union, then U. intersection: empty set.
% if pos approximate set and f union, f(f(...f(A',B'))) = U, intersection -> AB.
%

\subsection{Random approximate Boolean search}
We consider an \emph{approximation} of the set-theoretic \emph{Boolean search} model where the Boolean search indexes are replaced by \emph{random approximate sets}, i.e., $\MakeIndex \colon \Set{D} \mapsto \PS{\Set{K}}$ is replaced with $\MakeIndex^{\sigma} \colon \Set{D} \mapsto \ASetType{\PS{\Set{K}}}$.
We denote the transformed find function by $\Find^\sigma$ as opposed to the objective function $\Find$.

This replacement \emph{induces} random approximate \emph{result sets} as given by the following theorem.
\begin{theorem}
$\Find^\sigma$ is an \emph{approximation} of $\Find$ where $\Find^\sigma(q,ds)$ is a random approximate set of $\Find(q,ds)$ for all $q$ in $Q$ and all $ds \in \PS{\Set{D}}$.
\end{theorem}
\begin{proof}
An approximate search index $\ASet{S}[\fnrate][\fprate]$ is relevant to a key $x$ if the key $x$ tests positive in it.
A false positive occurs if the key $x$ is not in $\Set{S}$ but is in $\ASet{S}$, which occurs with some probability $\fprate \geq 0$.
A false negative occurs if a key $x$ is in $\Set{S}$ but is not in $\ASet{S}$, which occurs with some probability $\fnrate \geq 0$.

We have established that the result sets of a single atomic key are approximate result sets. We may now apply the set-theoretic results in \cref{dummyref} to implement the full set-theoretic model for approximate sets.

Continue proof here.
\end{proof}

$\Find^\sigma$ is a function and therefore produces the same output (result set) for the same input (query).
However, it still obeys the axioms of the random approximate set model since as described in \cref{dummyref}. 

\begin{example}
Suppose the search indexes are positive approximate sets each with a false positive rate $\fprate$. A common type of Boolean query is the \emph{intersection} (conjunction) of atomic keys. Consider a conjunctive query of $k$ keys, $x_1,\ldots,x_k$. The result set $\PASet{R}\left(\{x_1\} \cap \cdots \cap \{x_k\}\right) = \Find\!\left(\SetComplement[\SetComplement[x_1] \cup \cdots \cup \SetComplement[x_k]]\right)$ is a positive approximate set with an uncertain false positive rate $[\fprate_k] = [\fprate^k, \fprate]$.
\begin{proof}
Let the approximate result set for key $x_j$ be denoted by $\PASet{R}[x_j]$. The result set is given by
\begin{equation}
    \PASet{R}[\Set{X}] = \bigcap_{j=1}^{k} \PASet{R}[j]\,.
\end{equation}
By \cref{cor:fprate_atomic_search}, $\PASet{R}[j]$ has a false positive rate $\fprate$ for $j \in [1,\ldots,k]$. By \cref{dummyref}, $\SetIntersection[\PASet{R}[1]][\PASet{R}[2]]$ has a false positive rate $[\fprate^2,\fprate]$. Similarly,
\begin{equation}
\SetIntersection[\left(\SetIntersection[\PASet{R}[1]][\PASet{R}[2]]\right)][\PASet{R}[3]] = \PASet{R}[1] \cap \PASet{R}[2] \cap \PASet{R}[3]    
\end{equation}
has a false positive rate $[\fprate^3,\fprate]$. Continuing in this fashion, we see that $\PASet{R}[\Set{X}] = \PASet{R}[1] \cap \cdots \cap \PASet{R}[k]$ has a false positive rate $[\fprate^k,\fprate]$.
\end{proof}
\end{example}

To quantify the performance measure of the information retrieval system, we may use the binary classification results in \cref{sec:perf}.
\end{document}