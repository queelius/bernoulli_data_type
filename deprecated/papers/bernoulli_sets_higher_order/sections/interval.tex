\documentclass[ ../main.tex]{subfiles}
\providecommand{\mainx}{..}
\begin{document}
\section{Uncertain rate distortions}
\label{sec:intervals}
We may not be certain about the \emph{expected} false positive and true positive rates, i.e., we may only have the joint distribution of $\ASet{A}$, $\FPR$, and $\FNR$.

\subsection{First-order model}
The easiest case to analyze is the \emph{first-order} random approximate set model.
Suppose we are interested in the distribution of $\ASet{A}$ given $\RV{R}$ has $p$ positives and $n$ negatives.
Since we are primarily interested in the distribution of false positives and true positives (or their corresponding rates), we consider the related random tuple $\Tuple{\FP_n, \TP_p, \TPR, \FPR}$ which, assuming $\FPR$ and $\TPR$ are independent, has a joint probability density function given by
\begin{equation}
	\PDF{t,f,\tprate,\fprate}[\TP_p, \FP_n,\TPR,\FPR] = \PDF{t, \tprate}[\TP_p,\TPR] \PDF{f,\fprate}[\FP_n,\FPR]
\end{equation}
where
\begin{align}
	\PDF{t,\tprate}[\TP_p,\TPR] &= \PDF{t \Given \tprate}[\TP_p \Given \TPR] \PDF{\tprate}[\TPR]\,,\\
	\PDF{f,\fprate}[\FP_n,\FPR] &= \PDF{f \Given \fprate}[\FP_p \Given \FPR] \PDF{\fprate}[\FPR]\,.
\end{align}

When we \emph{marginalize} over the true positives, we get the result
\begin{equation}
\begin{split}
	\PDF{t}[\TP_p] &= \int_{0}^{1} \PDF{t \Given \tprate}[\TP_p \Given \TPR] \PDF{\tprate}[\TPR] \dif \tprate\\
	                 &= \int_{0}^{1} \binom{p}{t} \tprate^{t}(1-\tprate)^{p-t} \PDF{\tprate}[\TPR] \dif \tprate\,.
\end{split}
\end{equation}

If all the probability mass for $\TPR$ is assigned to a particular point $\tprate$, the probability mass function simplifies to
\begin{equation}
\PDF{t}[\TP_p] = \binom{p}{t} \tprate^{t_p}(1-\tprate)^{p-t}\,,
\end{equation}
which is probability mass function of a binomial distribution.

\begin{figure}
	\def\svgwidth{\columnwidth}
	%\def\svgscale{0.75}
	\centering
	\captionsetup{justification=centering}
	\caption
	{
		Relative frequency of positive predicitive values for several different parameterizations of the false positive and true positive rates given $\n = 900$ negatives and $\p=100$ positives.
	}    
	\input{img/norms.pdf_tex}
	\label{fig:mixnorm}
\end{figure}

The simplest kind of uncertainty is given by a disjoint set of intervals, in which the true expected rate is uniformly distributed across the support.
\begin{definition}
An interval is a convex set of real numbers. We denote by $\Interval{x} = \Interval{\underbar{x},\bar{x}}$ an interval with a lower-bound $\underbar{x}$ and an upper-bound $\bar{x}$.
\end{definition}

A further simplification comes from mapping the disjoint set of intervals to the smallest interval that \emph{spans} all of them.
\begin{definition}
Given a disjoint interval set $\Set{X}$, $\IntervalSpan(\Set{X})$ maps to an interval with lower and upper bounds that are the lower and upper bounds of $\Set{X}$.
\end{definition}




A confidence interval, for instance, may be specified in this notation.
Here, however, we consider an algebra for interval arithmetic and put it to use quantifying our ignorance about the distribution of parameters after, for instance, a union operation.

The performance measures summarized by \cref{tbl:perf_sum} depend upon the false positive rate $\fprate$, false negative rate $\fnrate$, and proportion of positives $\lambda$ being \emph{known}.
Any parameters that are not known with certainty may be replaced in the above table by intervals that (are assumed to) contain the expected value.
As a consequence, the performance measure will also be an interval.

\emph{Maximum} uncertainty is when the parameter value is in the interval $[0,1]$, e.g., $[\lambda] = [0,1]$, and \emph{minimum} uncertainty is when the parameter is some value in the degenerate interval $[x,x]$, e.g., $[\fprate] = [.2,.2]$. The more certain--the smaller the width of the intervals--the more certain the performance measure.

When using interval arithmetic, the \emph{dependency problem} can lead to overly pessimistic bounds.
In our case, the formulae are simple enough to ensure dependencies are satisfied. We show the results of an uncertain proportion of positives $[\lambda]$ for the \emph{accuracy} measure in the following example.

\begin{example}
	Suppose we wish to determine the expected accuracy given that the proportion of positives is known to be some value in the interval 
	$[\lambda]$. Then, the expected accuracy is some value in the interval
	\begin{equation}
		\begin{split}
		\acc([\fprate],[\fnrate] ; [\lambda]) =
		\biggl[
		\operatorname{f}(\overline{\fprate},\overline{\fnrate})(1-\overline{\fnrate}) + 
		\left(1-\operatorname{f}(\overline{\fprate},\overline{\fnrate})\right)(1-\overline{\fprate}),\\
		\operatorname{f}(\underline{\fnrate},\underline{\fprate})(1-\underline{\fnrate}) + 
		\left(1-\operatorname{f}(\underline{\fnrate},\underline{\fprate})\right)(1-\underline{\fprate})
		\biggr]\,,
		\end{split}
		\end{equation}
		where $\operatorname{f}(x,y) = \overline{\lambda}[x < y] + 
		\underline{\lambda}[y \leq x]$. If we have complete ignorance about 
		$\lambda$ then $[\lambda] = [0,1]$. As a special case, if we have 
		complete 
		ignorance about lambda and $\fnrate = 0$ (positive approximate set), 
		then 
		$\acc([\fprate],0 ; [0,1]) = [1-\overline{\fprate}, 1]$.
\end{example}

However, the expected rates may not be known, e.g., the values of $\alpha_1$ and $\alpha_2$ in \cref{eq:union_fn_rate} may not be known.
Alternatively, we may not be interested in the \emph{expected} value, but the smallest set of values such that with probability $1-\alpha$ the true rate 
realizes some value in the set, which is typically an \emph{interval}, i.e., a confidence interval.

Intervals represent an uncertainty and they manifest themselves in two independent ways.
The common notion of the \emph{confidence interval} is a product of the probabilistic model, i.e., the realized true positive rate $\tprateob$, which is normally centered around the expected true positive rate $\tprate$ as discussed in \cref{sec:asymtotic}.
We may use \emph{interval arithmetic} and replace point values interval values, point values being a degenerate case.
Basic interval arithmetic is presented in \cite{basicinterval}.

A set sampled from $\ASet{A}(\fprate,\tprate)$ is an approximate set such that 
the $(1-\alpha)\%$ asymptotic confidence interval for the false negative and 
false positive rates given respectively by
\begin{equation}
\label{eq:fnrate_union_set}
\Interval{\fnrate} = ?
\end{equation}
and
\begin{equation}
\label{eq:fprate_union_set}
\Interval{\fprate} = ?\,.
\end{equation}


By \cref{dummyref}, $\SetUnion[\ASet{A}][\ASet{B}]$, the observation 
$\ASet{A}(\fprate,\tprate) = \Set{A}$ is an approximate set with a false 
negative rate
\begin{equation}
\label{eq:fnrate_union_set}
\fnrateob \in \Interval{\fnrate_1}(1 - \Interval{\fprate_2}) \cup  
\Interval{\fnrate_2} 
(1 - \Interval{\fprate_1}) \cup [\fnrate_1][\fnrate_2]
\end{equation}
and a false positive rate
\begin{equation}
\label{eq:fprate_union_set}
\fprateob \in 1 - (1 - \Interval{\fprate_1})(1 - \Interval{\fnrate_2})\,.
\end{equation}

\Cref{eq:fnrate_union_set} represents a disjoint set of intervals.
However, we  are only interested in the best and worst case of the false negative rate.
Thus, we map the disjoint set to a \emph{minimum width} interval that contains every point in the disjoint set.
\begin{definition}
	Given a set $\Set{X}$, $\IntervalSpan(\Set{X})$ maps to an interval with 
	lower and upper bounds that are the lower and upper bounds of $\Set{X}$.
\end{definition}

\begin{theorem}
	\label{thm:uncertain_rates_union_set}
	The \emph{union} of two approximate sets with uncertain false negative 
	rates 
	$\Interval{\fnrate_1}$ and $\Interval{\fnrate_2}$ and uncertain false 
	positive rates $\Interval{\fprate_1}$ and $\Interval{\fprate_2}$ is an 
	approximate set with an uncertain false negative rate
	\begin{equation}
	\label{eq:fnrate_union_set}
	\begin{split}
	\Interval{\fnrate}
	&= \IntervalSpan([\fnrate_1](1 - \Interval{\fprate_2}) \cup  
	[\fnrate_2] (1 - \Interval{\fprate_1}) \cup 
	\Interval{\fnrate_1}[\fnrate_2])\\
	&= \biggl[
	\min\!\left\{
	\underline{\fnrate}_1(1 - \overline{\fprate}_2),
	\underline{\fnrate}_2(1 - \overline{\fprate}_1),
	\underline{\fnrate}_1 \underline{\fnrate}_2
	\right\},\\
	& \qquad \qquad \max\!\left\{
	\overline{\fnrate}_1(1 - \underline{\fprate}_2),
	\overline{\fnrate}_2(1 - \underline{\fprate}_1),
	\overline{\fnrate}_1 \overline{\fnrate}_2\right\}
	\biggr]
	\end{split}
	\end{equation}
	and an uncertain true negative rate
	\begin{equation}
	\label{eq:fprate_union_set}
	\begin{split}
	\Interval{\tnrate}
	&= \Interval{\tnrate_1}\Interval{\tnrate_2}\\
	&= \Interval{
		\underline{\tnrate}_1 \underline{\tnrate}_2,
		\overline{\tnrate}_1 \overline{\tnrate}_2}\\
	&= 1-(1-\Interval{\fprate_1})(1-\Interval{\fprate_2})\\
	&= \Interval{
		\underline{\fprate}_1 + \underline{\fprate}_2 - 
		\underline{\fprate}_1 \underline{\fprate}_2\,, \overline{\fprate}_1 
		+ \overline{\fprate}_2 - \overline{\fprate}_1 \overline{\fprate}_2}\,.
	\end{split}
	\end{equation}
\end{theorem}
\begin{proof}
	By \cref{dummyref}, the false positive rate of $\ASet{A} \cup \ASet{B}$ is
	\begin{equation}
	\left[\fprate\right] = \left[\fprate_1\right] + \left[\fprate_2\right] - 
	\left[\fprate_1\right] \left[\fprate_2\right]\,.
	\end{equation}
	and the false negative rate is
	\begin{equation*}
	\begin{split}
	\left[\fnrateob\right] &=
	\alpha_1 \left[\fnrateob_1\right](1 - \left[\fprate_2\right]) + 
	\alpha_2 \left[\fnrateob_2\right] (1 - \left[\fprate_1\right])\\
	&\qquad+ (1 - \alpha_1 - \alpha_2) \left(1 - \left[\fprate_1\right] + 
	\left[\fnrateob_2\right] \left[\fprate_1\right]\right)\,,
	\end{split}
	\tag{\ref{eq:union_fn_rate} revisited}
	\end{equation*}
	where $\alpha_1,\alpha_2 \geq 0$ and $\alpha_1 + \alpha_2 \leq 1$. Thus, to 
	maximize (minimize) this equation, we simply need to put all of the 
	\emph{weight} into the largest (smallest) term.
\end{proof}


\begin{theorem}
	\label{thm:uncertain_rates_comp_set}
	The \emph{complement} of an approximate set with a false negative rate $\Interval{\fnrate}$ and false positive rate $\Interval{\fprate}$ is an approximate set 
	with a false negative rate $\Interval{\fprate}$ and false positive rate $\Interval{\fnrate}$.
\end{theorem}

Since any set-theoretic composition is reducible to a combination of unions and 
complements, we may use 
\cref{thm:uncertain_rates_union_set,thm:uncertain_rates_comp_set} to compute 
the bounds for any set-theoretic composition of approximate sets. See 
\cref{tab:set_theory_intervals} for a summary of a several well-known 
operations.

\begin{table}[h]
	\centering
	\caption{The smallest intervals that contain the false positive and false 
		negative rates of the approximate sets that result from the 
		corresponding 
		set-theoretic operations on approximate sets 
		$\ASet{A}\left(\Interval{\fnrate_1},\Interval{\fprate_1}\right)$ and 
		$\ASet{B}\left(\Interval{\fnrate_2},\Interval{\fprate_2}\right)$.}
	\label{tab:bin_op_interval}    
	\begin{tabular}{@{} l l l @{}}
		\toprule
		\textbf{op} & \textbf{param} & \textbf{interval}\\
		\midrule
		$\SetUnion[\ASet{A}][\ASet{B}]$ & $\left[\fprate\right]$ &
		$1 - (1 - \Interval{\fprate_1})(1 - \Interval{\fprate_2}$\\ 
		\addlinespace
		&$\Interval{\fnrate}$ &
		$\IntervalSpan(\Interval{\fnrate_1}(1 - \Interval{\fprate_2}) \cup  
		\Interval{\fnrate_2} (1 - 
		\Interval{\fprate_1}) \cup \Interval{\fnrate_1}[\fnrate_2])$\\ 
		\cmidrule{1-3}
		$\SetIntersection[\ASet{A}][\ASet{B}]$ & $\left[\fprate\right]$ & 
		$\IntervalSpan(\Interval{\fnrate_1}(1 - \Interval{\fnrate_2}) \cup  
		\Interval{\fprate_2} (1 - 
		\Interval{\fnrate_1}) \cup \Interval{\fnrate_1}\Interval{\fprate_2})$\\
		\addlinespace
		&$\Interval{\fnrate}$ &
		$1 - (1 - \Interval{\fnrate_1})(1 - \Interval{\fnrate_2}$\\ 
		\cmidrule{1-3}
		$\SetDiff[\ASet{A}][\ASet{B}]$ &
		$\Interval{\fprate}$ &
		$\IntervalSpan(\Interval{\fnrate_1}(1 - \Interval{\fprate_2}) \cup  
		\Interval{\fnrate_2}(1 - 
		\Interval{\fnrate_1}) \cup \Interval{\fnrate_1}\Interval{\fnrate_2})$\\
		\addlinespace
		&$\Interval{\fnrate}$ &
		$\Interval{
			\underline{\fnrate}_1 + \underline{\fprate}_2(1 - 
			\overline{\fnrate}_1),
			\overline{\fnrate}_1 + \overline{\fprate}_2(1 - 
			\underline{\fnrate}_1)            
		}$\\
		\cmidrule{1-3}
		$\SetComplement[\ASet{A}]$ & $\left[\fprate\right]$ & 
		$\Interval{\fnrate_1}$\\
		&$\Interval{\fnrate}$ &  $\Interval{\fprate_1}$\\
		\bottomrule
	\end{tabular}
\end{table}

\begin{table*}
	\caption{The tightest intervals that contain the false positive and false 
	negative rates of the positive or negative approximate sets that result 
	from the corresponding set-theoretic operations.}
	\label{tab:neg_pos}    
	\begin{subfigure}[b]{\columnwidth/2}
		\centering
		\caption{$\PASet{A}(\Interval{\fprate_1})$ and 
		$\PASet{A}(\Interval{\fprate_2})$.}
		\begin{tabular}{@{} l l l @{}}
			\toprule
			\textbf{op} & \textbf{param} & \textbf{interval}\\
			\midrule
			$\SetUnion[\PASet{A}][\PASet{B}]$ & $\Interval{\fprate}$ &
			$1 - \left(1 - \Interval{\fprate_1}\right)\left(1 - 
			\Interval{\fprate_2}]\right)$\\ 
			\cmidrule{1-3}
			$\SetIntersection[\PASet{A}][\PASet{B}]$ & $\Interval{\fprate}$ 
			&
			$\left[
			\underline{\fprate}_1 \underline{\fprate}_2, 
			\max\!\left(\overline{\fprate}_1,\overline{\fprate}_2\right)
			\right]$\\
			\cmidrule{1-3}
			$\SetDiff[\PASet{A}][\PASet{B}]$ & $\Interval{\fprate}$ &
			$\left[0,\overline{\fprate}_1(1-\overline{\fprate}_2\right]$\\ 
			\addlinespace
			&$\Interval{\fnrate}$ & $\Interval{\fprate_2}$\\
			\cmidrule{1-3}
			$\SetComplement[\PASet{A}]$ & $\Interval{\fnrate}$ &  
			$\Interval{\fprate_1}$\\
			\bottomrule
		\end{tabular}
	\end{subfigure}
	\begin{subfigure}[b]{\columnwidth/2}
		\centering
		\caption{$\NASet{A}(\Interval{\fnrate_1})$ and 
		$\NASet{B}(\Interval{\fnrate_2})$.}
		\begin{tabular}{@{} l l l @{}}
			\toprule
			\textbf{op} & \textbf{param} & \textbf{interval}\\
			\midrule
			$\SetUnion[\NASet{A}][\NASet{B}]$ & $\Interval{\fnrate}$ &
			$\left[\underline{\fnrate}_1 \underline{\fnrate}_2,
			\max\!\left(\overline{\fnrate}_1,
			\overline{\fnrate}_2\right)
			\right]$\\ 
			\cmidrule{1-3}
			$\SetIntersection[\NASet{A}][\NASet{B}]$ & $\Interval{\fnrate}$ &
			$1 - (1 - \Interval{\fnrate_1})(1 - \Interval{\fnrate_2})$\\ 
			\cmidrule{1-3}
			$\SetDiff[\NASet{A}][\NASet{B}]$ & $\Interval{\fprate}$ &
			$\left[
			0,
			\overline{\fnrate}_2(1 - \underline{\fnrate}_1)\right]$\\ 
			\addlinespace
			&$\Interval{\fnrate}$ &
			$\Interval{\fnrate_1}$\\
			\cmidrule{1-3}
			$\SetComplement[\NASet{A}]$ & $\Interval{\fprate}$ & 
			$\Interval{\fnrate_1}$\\
			\bottomrule
		\end{tabular}
	\end{subfigure}
\end{table*}

% TODO: is the interval a max, as claimed, or a min, i.e, max(e1,e2) or 
%min(e1,e2)? in the asymptotic limit
% we claim that infinite intersections of A(ei) converges almost surely to A. 
%but if max, then by the formula
% it converges to A(e = [0, max(e1,e2,...,en)]... maybe its min?


\begin{example}
Suppose we have three sets $\Set{A}$, $\Set{B}$, and $\Set{C}$ and consider the random approximate set
\begin{equation}
	\ASet{D}[\fprate][\tprate] = 
	\SetDiff[
	\left(
	\SetIntersection[\PASet{A}[\fprate]][\PASet{B}[\fprate]]
	\right)
	]
	[\PASet{C}[\fprate]]\,.
\end{equation}
The intersection of $\PASet{A}$ and $\PASet{B}$ is an approximate set 
\begin{equation}
	\SetIntersection[\PASet{A}][\PASet{B}][\Interval{\fprate}] =
	\SetComplement[\SetUnion[\SetComplement[\ASet{A}]][\SetComplement[\ASet{B}]]]\,.
\end{equation}
\end{example}













\end{document}