\documentclass[ ../main.tex]{subfiles}
\providecommand{\mainx}{..}

\SetKwData{Nothing}{nothing}

\begin{document}
\section{C++ implementation}



In programming languages, \emph{composite} types are typically composed in two ways, the \emph{sum type} and the \emph{product type}.
The product type is a Cartesian product of types (sets).

For instance, a product type may the Cartesian product of integer and Boolean types, $\IntSet \times \{0,1\}$. In the C++ family of programming language, this product type may be represented by \emph{pair<int,bool>}.

Since it may be inconvenient to refer to member types by their respective tuple indices, programming languages typically allow the indices to be \emph{labeled}. In C++, keywords like \emph{struct} and \emph{class} are commonly used to provide named product types.

A data type may be thought of as a product type with \emph{invariants}, or \emph{constraints} on which values member types may be assigned. Thus, it may be thought of as a \emph{relation} or \emph{correspondence} over the product type. In many cases, this may make approximate relations unsuitable for representing data types, e.g., a tuple that violates one or more of the invariants may be generated from an objective value of the data type.

However, if an approximate relation does violate the invariants, it can simply be considered invalid. In this case, we may pose questions like, what is the probability that an approximate data type generates an invalid result?

%An algebraic data type is defined by a set of \emph{higher-order} operations on \emph{data types}.
%In other words, operations on sets, like Cartesian product or disjoint union.
Viewing a \emph{type} as a set, most programming languages have primitive types like \emph{integers}, \emph{Booleans}, and \emph{characters}.
In C++, these are respectively denoted by \mintinline{c++}{int}, \mintinline{c++}{bool}, and \mintinline{c++}{char} with cardinalities given respectively by $2^{32}$, $2^8$, and $2^1$.
Any type needs one or more \emph{value constructors} to construct objects that model values in that type.
For instance, in C++ the value that denotes the Boolean value of \emph{true} is constructed with the syntax \mintinline{c++}{true}.

The \emph{unit} type is special singleton set with a single value, i.e., a cardinality of $2^0$.
In C++, the confusing notation of \mintinline{c++}{void} denotes the unit type (and the single value).
\begin{remark}
	The \emph{absurd} type is a special type with zero values, i.e., the \emph{empty set}.
	Since there are no values in the absurd type, no values of this type can be constructed.
	There is no primitive absurd type in C++.
\end{remark}

A \emph{product type} is the $n$-fold Cartesian product of zero or more types where the zero-th product is the unit type.
For instance, in C++, \mintinline{c++}{struct { char y, bool z }} is a \emph{named} product type and \mintinline{c++}{tuple<char,bool>}
is the \emph{unamed} counterpart, where both are product types \mintinline{c++}{char} $\times$ \mintinline{c++}{bool}.
The cardinality of this product type is $2^8 \cdot 2^1 = 512$.
One way a particular value of this product type may be constructed is \mintinline{c++}{tuple<char,bool>('a', true)}.

The values of a sum type are typically grouped into several classes, called variants.
The set of all possible values of a sum type is the disjoint union of the sets of all possible values of its variants.
For instance, in C++, \mintinline{c++}{variant<char,bool>}
is the sum type \mintinline{c++}{char} $+$ \mintinline{c++}{bool}, which has a cardinality of $2^8 + 2^1 = 258$.
One way a particular value of this sum type may be constructed is \mintinline{c++}{variant<char,bool>('1', true)}.
A particularly useful type in C++ is \mintinline{c++}{optional<X>}, which conceptually models the sum type \mintinline{c++}{X} + \mintinline{c++}{void} where \mintinline{c++}{void} denotes the variant ``not a value of type \mintinline{c++}{X}.'
\begin{remark}
	This is not valid C++ syntax, even though the unit type value should be first-class.
\end{remark}
The cardinality of the \mintinline{c++}{optional<X>} is the cardinality of \mintinline{c++}{X} plus $1$.
We label the value in this singleton \emph{nothing} and may test whether a particular value is either \emph{nothing} or alternatively a value in \mintinline{c++}{X}.

Exponential types are \emph{functions}.
In C++, \mintinline{c++}{[](tuple<char,bool>) -> bool} is the set of functions \mintinline{c++}{char} $\times$ \mintinline{c++}{bool} $\mapsto$ \mintinline{c++}{bool}, which has a cardinality of $2^{512}$.
Usually, a more convenient syntax is used, like \mintinline{c++}{[](char,bool) -> bool}.
The constant \mintinline{c++}{true} function of the exponential type \mintinline{c++}{char} $\times$ \mintinline{c++}{bool} $\mapsto$ \mintinline{c++}{bool} may be constructed with the definition \mintinline{c++}{[](char,bool) -> bool { return true; }}.
\begin{remark}
	The exponential type \mintinline{c++}{[](X x) -> void} is of little practical value and, in C++, usually denotes a \emph{procedure} that causes \emph{side-effects} like writing to IO.
\end{remark}

Recall that any subset of a set corresponds to a \emph{relation}. Types are \emph{subsets} of the algebraic types where subsets are defined by \emph{invariants}.
We may compose primitive types to specify a variety of \emph{compound} types.
\begin{example}
	\emph{Rationals} may be implemented as a product type of two integers,
	\begin{minted}{c++}
		using Rational = tuple<int,int>,
	\end{minted}
	where the first and last elements of the tuple represent the numerator and denominator respectively.
	If the invariant is that the denominator is not $0$, then a \emph{value constructor} \mintinline{c++}{rational} $\colon$ \mintinline{c++}{int} $\times$ \mintinline{c++}{int} $\mapsto$ \mintinline{c++}{optional<Rational>} that takes a numerator and denominator and outputs either a rational or, if the invariants are violated, \emph{nothing}, is given by \cref{lst:valuector}.\footnote{Alternatively, the value constructor can be a \emph{partial function} \mintinline{c++}{rational} $\colon$ \mintinline{c++}{int} $\times$ \mintinline{c++}{int} $\pfun$ \mintinline{c++}{Rational} that is undefined for input $\Pair{x}{0}$ for any $x \in$ \mintinline{c++}{int}.}
	
%\begin{listing}[H]
\begin{minted}
	[
	mathescape=true,
	%	frame=single,
	%	frame=lines,
	%frame=leftline,
	framesep=2mm,
	%	baselinestretch=.9,
	%	bgcolor=LightGray,
	%	fontsize=\footnotesize,
	%	linenos
	]{c++}
optional<Rational> rational(int num, int denom)
{
	if (denom == 0)
	return {}; // Return the value that denotes $\Nothing$.  
	return tuple<int,int>(num, denom);
}.
\end{minted}
%\caption{A value constructor for rationals.}
%\label{lst:valuector}
%\end{listing}

	The expected operators on rationals, like addition \mintinline{c++}{operator+(Rational,Rational) -> Rational}, may be implemented so that \mintinline{c++}{Rational} models the \emph{concept} of rationals.
	%As long as we use \emph{regular} functions over a regular syntax, we may substitute any rational implementation for any other to produce equivalent behavior on any algorithm.
	
	The \mintinline{c++}{Rational} type is a subset of the product type \mintinline{c++}{tuple<int,int>} and is thus
	a \emph{binary relation} on $\IntSet \times \IntSet$.
	\begin{remark}
		We implemented \mintinline{c++}{Rational} as a product type \mintinline{c++}{tuple<int,int>} and a value constructor \mintinline{c++}{rational} for pedagogical reasons, but generally programming structures like \emph{class} are utilized since they facilitate important concepts like \emph{encapsulation}.
	\end{remark}
\end{example}

Each of these types and operators has a corresponding random approximation, e.g., the exponential type is just a random approximate map, and the \emph{relations} that define types are just random approximate relations with \emph{deterministic} properties that model the invariants.

The invariants may not be easy to satisfy, and so a random approximation relation of the corresponding type may not be practical.
However, when the invariants can be satisfied, we may implement \emph{random approximate algebraic data types} of the \emph{algebraic data types}, e.g., we can compose random approximate algebraic data types as before to construct compound random approximate data types of the corresponding compound algebraic data type.

This may not seem particularly useful, but it permits space-efficient representations and, moreover, concepts like \emph{oblivious algebraic data types} may be based on it with some notion of closure.

\subsection{Approximate value monad}

Talk about lifting functions.




Suppose we have a function $\Fun{f} \colon X \mapsto Y$.



 and we apply $\Fun{f}$ to an approximate value of type $X$, denoted by $\Fun{a}(X)$, 


$\Fun{f}$  $\Fun{a}(\Fun{f}) \colon X \amapsto{\fprate}{\fnrate} Y$.

The approximate value parameterized by $X$, denoted by $\mintinline{c++}{approx<X>}$, is a monad type similar to the \emph{maybe} monad discussed previously, except significantly more complicated if \emph{fully} implemented.\footnote{We may choose the trivial implementation that just tags it as an approximate value and let that bit of information carry through.}

The general type of the approximate value monad is simply defined as
\begin{minted}{c++}
	template <typename X> struct approx<X> {}
\end{minted}
which has a computational basis given by the following overload set.

Given an approximate value of type $X$, its \emph{false positive rate} is given by
\begin{minted}{c++}
	template <typename X>
	auto fpr(approx<X> x)   { /* implementation */ },
\end{minted}
its \emph{false negative rate} is given by
\begin{minted}{c++}
	template <typename X>
	auto fnr(approx<X> x)   { /* implementation */ },
\end{minted}
its value is given by
\begin{minted}{c++}
	template <typename X>
	auto value(approx<X> x) { /* implementation */ },
\end{minted}
and its conditional probability mass function is given by
\begin{minted}{c++}
	template <typename X>
	auto pmf(approx<X> x, X true_value) { /* implementation */ }.
\end{minted}

Note that the false positive and false negative rates are the result of using the distance function $\Fun{d}(a,a) = 0$ and otherwise $\Fun{d}(a,b) = 1$, $a \neq b$, to calculate the \emph{expected} loss given respectively the negative (where $\Fun{d}$ evaluates to $1$ with respect to some ground truth) and positive (where $\Fun{d}$ evaluates to $0$) subsets of $X$.

If we give it a \emph{loss} function, we may estimate its loss.


For particular value types, we specialize this template.
\begin{minted}{c++}
	template <> struct approx<bool>
	{
		double fpr;
		double fnr;
		bool value;
	}
\end{minted}
which has a computational basis given by
\begin{minted}{c++}
	auto fpr(approx<bool> x)   { return x.fpr; },
	auto fnr(approx<bool> x)   { return x.fnr; },
	auto value(approx<bool> x) { return x.value; }.
\end{minted}


The function $\mintinline{c++}{fmap} \colon (\mintinline{c++}{bool} \mapsto \mintinline{c++}{bool}) \times \mintinline{c++}{approx<bool>} \mapsto \mintinline{c++}{approx<bool>}$ is defined as
\begin{minted}{c++}
auto amap(function<bool(bool)> f, approx<bool> x)
{
	auto fnr = f(true);
	auto fpr = f(false);

	return approx<bool> { fpr, fnr, f(value(x)) };
}
\end{minted}



If we compose functions to construct an \emph{algorithm} (or program), then if some of the values (e.g., functions or Boolean values) are replaced by approximate values, the algorithm becomes (in general) an approximate algorithm and we may deduce its false positive and negative rates as we did previously through, say, function composition.

If the algorithm includes \emph{branching}, the algorithm is still an approximate algorithm, but in order to infer the false positive and false negative rate, we must go down \emph{all} branches, which is in general not be tractable.
However, we could estimate the result using \emph{Monte-carlo} simulation.





The approximate Boolean value is straightforward since Boolean types are the natural predicate return type.
More importantly, there are only two values of this type.


%template<class A, class B>
%optional<B> fmap(std::function<B(A)> f, %optional<A> opt) {
%	if (!opt.isValid())
%	return optional<B>{};
%	else
%	return optional<B>{ f(opt.val()) };
%}

	
\subsection{Type-erasure}
The fact that we are dealing with specific approximate types may be erased into some approximate type abstract data type that hides the concrete data type.
It may be further erased to just a type, e.g., $\APFun{f}[\fprate][\fnrate]$ may be erased to $\APFun{f}$ which may be erased to $\Fun{f}$.
TODO: convert to types not particular elements in the type
	
\end{document}