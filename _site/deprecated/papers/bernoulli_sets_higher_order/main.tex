%\documentclass[11pt,extrafontsizes,final,oneside,openany,hidelinks]{memoir}
%\documentclass[10pt,extrafontsizes,final,hidelinks]{article}
\documentclass[10pt,final,hidelinks]{article}
%\documentclass[final,hidelinks,twocolumn]{article}
%\pagestyle{headings}

\usepackage{lmodern}
\usepackage[T1]{fontenc}
\usepackage[english]{babel}
\usepackage[utf8]{inputenc}
\usepackage[activate={true,nocompatibility},final,tracking=true,kerning=true,
    spacing=true,factor=1100,stretch=10,shrink=10]{microtype}
\microtypecontext{spacing=nonfrench}
\usepackage[margin=.5in]{geometry}
\usepackage{graphicx}
\graphicspath{{img/}} 
\usepackage[noend]{algorithm2e}
\usepackage{caption}
\usepackage{subcaption}
\usepackage{booktabs}
\usepackage{array}
\usepackage{mathtools}
\usepackage{subfiles}
\usepackage{enumitem}
\usepackage{commath}
\usepackage{appendix}
\renewcommand\appendixtocname{Appendices}
\usepackage{amsmath}
\usepackage{amsthm}
\usepackage{amssymb}
%\usepackage[intoc,english]{nomencl}
%\makenomenclature
\usepackage{pgfplots}
\usepackage{tabu} 
\usepackage{tikz}
\usepackage{tikzscale}
\usepackage[square,numbers]{natbib}
\bibliographystyle{plain}
\usepackage{siunitx}
\numberwithin{equation}{section}
\usepackage{hyperref}
\usepackage{cleveref}

%\setlrmarginsandblock{1in}{1in}{*}
%\setulmarginsandblock{1in}{1in}{*}
%\checkandfixthelayout 

\newcommand{\COMMENT}[1]{}

%\usepackage{alexmisc}
\usepackage{functionnotation}
\usepackage{algorithmnotation}
\usepackage{matrixnotation}
\usepackage[fancy]{setnotation}
\usepackage[fancy]{relationnotation}
\usepackage{approxsetnotation}
\usepackage{approxrelationnotation}
\usepackage[section]{envnotation}

\COMMENT{
% Code for nomenclature
\renewcommand{\nompreamble}{The following nomenclature describes several symbols and notations in use.}
% This code creates the groups
% -----------------------------------------
\usepackage{etoolbox}
\renewcommand\nomgroup[1]{%
	\item[\bfseries
	\ifstrequal{#1}{R}{Random approximate sets}{%
		\ifstrequal{#1}{S}{Classical sets}{%
			\ifstrequal{#1}{T}{Set theory}{%
				\ifstrequal{#1}{P}{Probability}{}}}}%
	]}
}

\hypersetup{
    pdftitle={A set-theoretic model of random approximate sets with derivations 
    of random approximate sets induced by set-theoretic operations and 
    corresponding binary classification measures modeled as random variables.},
    pdfauthor={Alexander Towell},               % author
    pdfsubject={computer science},              % subject of the %document
    pdfkeywords={
        probabilistic data structure,
        abstract data type,
        approximate set,
        bloom filter,
        perfect hash function,
        perfect hash filter},                   % keywords
    colorlinks=true,                            % false: boxed links;
    linkcolor=magenta,
    citecolor=green,                            % color of links to 
    filecolor=blue,                             % color of file links
    urlcolor=green                              % color of external
}

\title
{
    An algebra of random approximate sets\\
    \large with derivations of higher-order random approximate sets induced by set-theoretic operations on random approximate sets with corresponding random binary classification measures.
}
\author
{
    Alexander Towell\\
    \texttt{atowell@siue.edu}
}
\date{}

\begin{document}
\maketitle
\begin{abstract}
We define a \emph{random approximate set} model and the probability space that 
follows.
A random approximate set is a \emph{probabilistic} set generated to \emph{approximate} another set of objective interest.
We derive several properties that follow from this definition, such as the expected \emph{precision} in information retrieval.
Finally, we demonstrate an application of approximate sets, approximate Encrypted Search with queries as a Boolean algebra, which generates random approximate result sets.
\end{abstract}

%\microtypesetup{protrusion=false}
\tableofcontents
%\microtypesetup{protrusion=true}
%\listoftables
%\listoffigures

%\nomenclature{$\vec{x}$}{Vectors are denoted in boldface font. The \jth component of $\vec{x}$ is denoted by $x_j$.}

\nomenclature[S]{$\Set{S}$}{Sets are normally notated like this.}
\nomenclature[S]{$\RealSet$}{The set of reals $(-\infty,\infty)$.}
\nomenclature[S]{$\NatSet$}{The set of natural numbers $\{0,1\}$.}
\nomenclature[S]{$\BitSet$}{The binary set $\{0,1\}$.}
\nomenclature[S]{$\Set{A}_{p}$}{The set $\SetBuilder{a \in \Set{A}}{\operatorname{p}(a)}$ where $\operatorname{p} \colon \Set{U} \mapsto \{0,1}$ is a predicate function over some universal set.}

\nomenclature[R]{$\ASet{A}$}{A \emph{random approximate set} of $\Set{A}$ where the false positive and negative rates are unspecified.}
%\nomenclature[R]{$\ASet{A}[\tprate][\fprate]$}{A \emph{random approximate set} of $\Set{A}$ where the parameterizations are $\tprate$ and $\fprate$.}
\nomenclature[R]{$\ASet{A}[\fnrate][\fprate]$}{A \emph{random approximate set} of $\Set{A}$ where the parameterizations are $\fnrate$ and $\fprate$.}
%\nomenclature[R]{$\ASet{A}[\tprate][+]$}{A \emph{random approximate set} of $\Set{A}$ where the false positive rate is unspecified or implicit.}
\nomenclature[R]{$\ASet{A}[\fnrate][+]$}{A \emph{random approximate set} of $\Set{A}$ where the false positive rate is unspecified.}
\nomenclature[R]{$\ASet{A}[-][\fprate]$}{A \emph{random approximate set} of $\Set{A}$ where the true positive rate is unspecified.}
\nomenclature[R]{$\ASet{A}[-][\tnrate]$}{A \emph{random approximate set} of $\Set{A}$ where the false negative rate is unspecified.}
\nomenclature[R]{$\PASet{A}$}{A \emph{positive} random approximate set of $\Set{A}$ where the false positive rate is unspecified.}
\nomenclature[R]{$\PASet{A}[\fprate]$}{A \emph{positive} random approximate set of $\Set{A}$ where the false positive rate is $\fprate$.}
\nomenclature[R]{$\NASet{A}$}{A \emph{negative} random approximate set of $\Set{A}$ where the true positive rate is unspecified.}
%\nomenclature[R]{$\NASet{A}[\tprate]$}{A \emph{negative} random approximate set of $\Set{A}$ where the true positive rate is $\tprate$.}
\nomenclature[R]{$\NASet{A}[\fnrate]$}{A \emph{negative} random approximate set of $\Set{A}$ where the false negative rate is $\fnrate$.}

\nomenclature[R]{$\fprate$}{By convention, an \emph{expected} false positive rate if nothing about it is known.}
\nomenclature[R]{$\tprate$}{By convention, an \emph{expected} true positive rate if nothing about it is known.}
\nomenclature[R]{$\tnrate$}{By convention, an \emph{expected} true negative rate if nothing about it is known.}
\nomenclature[R]{$\fnrate$}{By convention, an \emph{expected} false negative rate if nothing about it is known.}

\nomenclature[R]{$\fprateob$}{By convention, an \emph{observed} false positive rate if nothing about it is known.}
\nomenclature[R]{$\tprateob$}{By convention, an \emph{observed} true positive rate if nothing about it is known.}
\nomenclature[R]{$\tnrateob$}{By convention, an \emph{observed} true negative rate if nothing about it is known.}
\nomenclature[R]{$\fnrateob$}{By convention, an \emph{observed} false negative rate if nothing about it is known.}

\nomenclature[T]{$\Card{\Set{A}}$}{The cardinality of a set $\Set{A}$.}
\nomenclature[T]{$\Set{A} \times \Set{B}$}{The Cartesian product of $\Set{A}$ and $\Set{B}$ defined as $\SetBuilder{\Pair{a}{b}}{a \in \Set{A} \land b \in \Set{B}}$.}
\nomenclature[T]{$\Set{B}^n$}{The Cartesian product $\Set{B} \times \cdots \times \Set{B}$, $n$ times.}
\nomenclature[T]{$\SetIndicator{}$}{The indicator function, $\SetIndicator{\Set{A}} \colon \Set{X} \mapsto \{0,1\}$ is defined by $\SetIndicator{\Set{A}}(x) = 1$ if $x \in \Set{A}$ and otherwise equals $0$.}
\nomenclature[T]{$\PowerSet(A)$}{The power set of $\Set{A}$.}
\nomenclature[T]{$\PS{A}$}{The power set of $\Set{A}$.}
\nomenclature[T]{$\SetUnion[\Set{A}][\Set{B}]$}{The union of $\Set{A}$ and $\Set{B}$.}
\nomenclature[T]{$\SetUnion[\ASet{A}][\ASet{B}]$}{The union of random approximate sets $\ASet{A}$ and $\ASet{B}$.}

\nomenclature[P]{$\PDF[\RV{X}]$}{The probability mass or density function of random variable $\RV{X}$.}
\nomenclature[P]{$\CDF[\RV{X}]$}{The cumulative distribution function of random variable $\RV{X}$.}

\nomenclature{$\BL$}{The bit length function.}
\nomenclature{$\RE$}{The relative efficiency function.}
\nomenclature{$\AbsoluteEfficiency$}{The absolute efficiency function.}


%\printnomenclature

\subfile{sections/intro}
\subfile{sections/algebra_of_sets}
\subfile{sections/aset_model}
\subfile{sections/derived_distributions}
\subfile{sections/derived_distributions_higher_order}
\subfile{sections/aset_theory}
\subfile{sections/interval}
\subfile{sections/aset_adt}
\subfile{sections/application_boolean_search}
\subfile{sections/appendix}

\bibliography{references}

\end{document}
