\documentclass[ ../main.tex]{subfiles}
\providecommand{\mainx}{..}
%\newcommand{\code}[1]{\texttt{#1}}
\newcommand{\tsr}{\tprate_{\ASet{X} \subseteq \ASet{Y}}}

\begin{document}	
%\appendixpage
\appendices
{
%\section{Additional proofs}
%In what follows, we show additional or more rigorous proofs that are not necessarily central to the paper but warrant mention.

\section{Proof of corollary~\ref{cor:fpr_as_vareps}}
\label{app:cor_fpr_as_vareps}
To say that the sequence $\FPR_1,\FPR_2,\ldots$ converges almost surely to $\fprate$ means that
\begin{equation}
\Prob{\lim _{\n \to \infty} \FPR_\n = \fprate} = 1\,.
\end{equation}

By \cref{cor:fpr_as_vareps}, given \emph{countably infinite} negatives, a random approximate set with a false positive rate $\fprate$ is \emph{certain} to obtain $\fprate$.
\begin{proof}
Hoeffding's inequality\cite{hoeffding} provides that $\FP_\n$ is concentrated around its mean $\n \fprate$ as given by
\begin{equation}
\Prob{(\fprate -\epsilon) \n \leq \FP_\n \leq (\fprate + \epsilon )\n}
\geq 1 - 2 \exp\!\left(-2\epsilon ^{2} \n\right)\,,
\end{equation}
where $\epsilon > 0$.
We are interested in the limiting probability
\begin{equation}
\begin{split}
\lim_{\n \to \infty} \Pr\!\left[(\fprate -\epsilon) \n \leq 
\FP_\n \leq (\fprate + \epsilon)\n\right] =\\
\lim_{\n \to \infty} \left\{ 1 - 
2 \exp\!\left(-2\epsilon ^{2} \n\right) \right\} = 1\,.
\end{split}
\end{equation}
As $\epsilon$ goes to $0$, $\lim_{\n \to \infty} \FP_\n$ converges almost surely to $\fprate \n$ and therefore $\lim_{\n \to \infty} \FP_\n / \n$ converges almost surely to $\fprate$.
\end{proof}

\section{Proof of theorem~\ref{thm:approx_expected_precision}}
\label{sec:proof_approx_expected_precision}
Given $\p$ positives and $\n$ negatives, by \cref{cor:approx_expected_precision} an approximate set with a false positive rate $\fprate$ and a false negative rate $\fnrate$ has an \emph{expected} precision given \emph{approximately} by
\begin{equation*}
    \ppv(\fnrate, \fprate ; \n, \p) \approx \frac{\overline{t}_p}{\overline{t}_p + \overline{f}_p} +
    \frac{\overline{t}_p \sigma_{\!f_p}^2 - \overline{f}_p \sigma_{\!t_p}^2}{\left(\overline{t}_p + \overline{f}_p\right)^3}\,,
    \tag{\ref{eq:cor_approx_expected_precision} revisited}
\end{equation*}
where $\overline{t}_p = \p \tprate$ is the \emph{expected} number of \emph{true positives}, $\overline{f}_p =  \n \fprate$ is the \emph{expected} number of \emph{false positives}, $\sigma_{\!t_p}^2 = \p \fnrate \tprate$ is the variance of the number of \emph{true positives}, and $\sigma_{\!f_p}^2 = \n \fprate \fnrate$ is the variance of the number of \emph{false positives}.
\begin{proof}
The positive predictive value is a random variable given by
\begin{equation}
    \frac{\TP_{\p}}{\TP_{\p} + \FP_{\n}}\,.
\end{equation}
We are interested in the \emph{expected} positive predictive value,
\begin{equation}
    \ppv(\fprate,\tprate) = \Expect{\frac{\TP_{\p}}{\TP_{\p} + \FP_{\n}}}\,.
\end{equation}
This expectation is of a non-linear function of random variables, which is problematic so we choose to approximate the expectation.

Let the \emph{positive predictive value} function be denoted by
\begin{equation}
    \operatorname{f}(t_p, f_p) = \frac{t_p}{t_p + f_p}\,,
\end{equation}
where $t_p$ is the number of true positives and $f_p$ is the number of false positives.
We approximate this function with a second-order Taylor series. The gradient of $\operatorname{f}$ is given by
\begin{equation}
    \nabla{\operatorname{f}}(t_p,f_p) =
    \frac{1}{(t_p + f_p)^2}
    \begin{bmatrix}
        f_p\\
        -t_p\\
    \end{bmatrix}
\end{equation}
and the Hessian of $\operatorname{f}$ is given by
\begin{equation}
    \mathcal{H}(t_p,f_p) =
    \frac{1}{(t_p + f_p)^3}
    \begin{bmatrix}
        -2 f_p & t_p-f_p\\
        t_p-f_p & 2 t_p \\
    \end{bmatrix}\,.
\end{equation}

A linear approximation $\operatorname{g}$ of $\operatorname{f}$ that is reasonably accurate near the expected value of $\TP_\p$, denoted by $\overline{t}_p$, and the expected value of $\FP_\n$, denoted by $\overline{f}_p$, is given by
\begin{equation}
    \operatorname{g}(t_p,f_p) =
    \operatorname{f}\left(\overline{t}_p,\overline{f}_p\right) + \nabla{\operatorname{f}}(\overline{t}_p,\overline{f}_p])^{\intercal}
    \begin{bmatrix}
        t_p - \overline{t}_p\\
        f_p - \overline{f}_p\\
    \end{bmatrix}
    + \frac{1}{2}
    \begin{bmatrix}
        t_p - \overline{t}_p\\
        f_p - \overline{f}_p\\
    \end{bmatrix}^{\intercal}
    \mathcal{H}(\overline{t}_p,\overline{f}_p)
    \begin{bmatrix}
        t_p - \overline{t}_p\\
        f_p - \overline{f}_p\\
    \end{bmatrix}\,.
\end{equation}
As a function of random variables $\TP_\p$ and $\FP_\n$, $\operatorname{g}\!\left(\TP_\p,\FP_\n\right)$ is a random variable. Since $\Expect{\TP_\p - \overline{t}_p} = 0$ and $\Expect{\FP_\p - \overline{f}_p} = 0$, we immediately simplify the expectation of $\operatorname{g}$ to
\begin{equation}
\label{eq:proof_hess1}
    \Expect{\operatorname{g}(\TP_\p,\FP_\n)} = \frac{\overline{t}_p}{\overline{t}_p+\overline{f}_p} + \frac{\Expect{A}}{(\overline{t}_p + \overline{f}_p)^3}
\end{equation}
where
\begin{equation}
    A = \frac{1}{2}
    \begin{bmatrix}
        \TP_\p - \overline{t}_p\\
        \FP_\n - \overline{f}_p
    \end{bmatrix}^{\intercal}
    \begin{bmatrix}
        -2 \overline{f}_p & \overline{t}_p-\overline{f}_p\\
        \overline{t}_p-\overline{f}_p & 2 \overline{t}_p
    \end{bmatrix}
    \begin{bmatrix}
        \TP_\p - \overline{t}_p\\
        \FP_\n - \overline{f}_p
    \end{bmatrix}\,.
\end{equation}
Multiplying the right column matrix by the Hessian matrix in $A$ results in
\begin{equation}
    A = \frac{1}{2}
    \begin{bmatrix}
        \TP_\p - \overline{t}_p\\
        \FP_\n - \overline{f}_p
    \end{bmatrix}^{\intercal}
    \begin{bmatrix}
        -2 \overline{f} \left(\TP_\p - \overline{t}_p\right) + \left(\overline{t}_p-\overline{f}_p\right)\left(\FP_\n - \overline{f}_p\right)\\
        \left(\overline{t}_p-\overline{f}_p\right)\left(\TP_\p - \overline{t}_p\right) + 2 \overline{t}_p\left(\FP_\n - \overline{f}_p\right)
    \end{bmatrix}
\end{equation}
Multiplying the left column matrix by the right column matrix in $A$ results in
\begin{equation}
    A = -\overline{f}_p \left(\TP_\p - \overline{t}_p\right)^2 + 
    \left(\overline{t}_p-\overline{f}_p\right)\left(\TP_\p - 
    \overline{t}_p\right)\left(\FP_\n - \overline{f}_p\right) + 
    \overline{t}_p\left(\FP_\n - \overline{f}_p\right)^2\,.
\end{equation}
As a linear operator, the expectation of $A$ is equivalent to
\begin{equation}
\label{eq:proof_of_exp_precision_approx_1}
    \Expect{A} = -\overline{f}_p \Expect{\TP_\p - \overline{t}_p}^2 + \left(\overline{t}-\overline{f}_p\right)\Expect{\left(\FP_\n - \overline{f}_p\right)\left(\TP_\p - \overline{t}_p\right)} + \overline{t}_p\Expect{\FP_\n - \overline{f}_p}^2\,.
\end{equation}
By definition, $\Expect{\TP_\p - \overline{t}_p}^2$ is the variance of $\TP_\p$, $\Expect{\FP_\n - \overline{f}_p}^2$ is the variance of $\FP_\n$, and $\Expect{\left(\FP_\n - \overline{f}_p\right)\left(\TP_\p - \overline{t}_p\right)}$ is the covariance of $\TP_\p$ and $\FP_\n$, which is $0$ since they are independent.
Making these substitutions results in
\begin{equation}
    \Expect{A} = \overline{t}_p\Var{\FP_\n} - \overline{f}_p \Var{\TP_\p}\,.
\end{equation}
Substituting this result into \cref{eq:proof_hess1} yields
\begin{equation}
\label{eq:proof_hess2}
    \Expect{\operatorname{g}(\TP_\p,\FP_\n)} = 
    \frac{\overline{t}_p}{\overline{t}_p+\overline{f}_p} + 
    \frac{-\overline{f}_p \Var{\TP_\p} + 
    \overline{t}_p\Var{\FP_\n}}{(\overline{t}_p + \overline{f}_p)^3}
\end{equation}
By \cref{thm:fpbinom}, $\FP_\n$ is binomially distributed with a mean $\n \fprate$ and a variance $\n \fprate \tnrate$ and by \cref{cor:tpbinom}, $\TP_\p$ is binomailly distributed with a mean $\p \tprate$ and a variance $\p \fnrate \tprate$.
\end{proof}

\section{Proof of theorem~\ref{thm:union_fn_rate}}
\label{sec:union_fn_rate}
Given $\ASet{A}(\fprate_1,\fnrate)$ and $\ASet{B}(\fprate_2,\fnrate_2)$, their union is an \emph{approximate set} with a false negative rate given by
\begin{equation*}
\begin{split}
    \fnrate &=
        \alpha_1 \fnrate_1(1 - \fprate_2) + \alpha_2 \fnrate_2(1 - \fprate_1)\\
        &\qquad+ (1 - \alpha_1 - \alpha_2) \fnrate_1 \fnrate_2\,,
\end{split}
\tag{\ref{eq:union_fn_rate} revisited}
\end{equation*}
where
\begin{equation*}
\tag{\ref{eq:union_fn_rate_alpha} revisited}
\begin{split}
   0 &\leq \alpha_1 = \frac{\Card{\SetDiff[\Set{S}[1]][\Set{S}[2]]}}{\Card{\SetUnion[\Set{S}[1]][\Set{S}[2]]}}\,,\\
   0 &\leq \alpha_2 = \frac{\Card{\SetDiff[\Set{S}[2]][\Set{S}[1]]}}{\Card{\SetUnion[\Set{S}[1]][\Set{S}[2]]}}\,,\\
   \alpha_1 + \alpha_2 &\leq 1\,.
\end{split}
\end{equation*}
\begin{proof}
Suppose we have two sets $\Set{S}[1]$ and $\Set{S}[2]$. The false negative rate $\fnrate$ is a probability conditioned on a positive in the union of sets $\Set{S}[1]$ and $\Set{S}[2]$ being a negative in the union of approximate sets $\ASet{S}[1]$ and $\ASet{S}[2]$.

The set of \emph{possible} false negatives is the set of positives, $\Set{S}[1] \! \cup \Set{S}[2]$, which is equivalent to the union of the disjoint sets
$\SetIntersection[\Set{S}[1]][\Set{S}[2]]$, $\SetDiff[\Set{S}[1]][\Set{S}[2]]$ and $\SetDiff[\Set{S}[2]][\Set{S}[1]]$.

The false negative rate is equivalent to ratio of the \emph{expected} number of false negatives to the maximum possible false negatives $\Card{\SetUnion[\Set{S}[1]][\Set{S}[2]]}$. Since they are disjoint, we may consider each independently to calculate the expected total number of false negatives.

Let $A_1$ denote the event $\RV{X} \in \ASet{S}[1]$, $A_2$ denote $\RV{X} \in \ASet{S}[2]$, $B_1$ denote $\RV{X} \in \Set{S}[1]$, and $B_2$ denote $\RV{X} \in \Set{S}[2]$. Suppose we randomly select an element from $\Set{S}[1] \! \cap \Set{S}[2]$. The probability that $\RV{X}$ is a negative in $\ASet{S}[1] \cup \ASet{S}[2]$ given that it is positive in $\Set{S}[1] \cap \Set{S}[2]$ is given by
\begin{equation}
    \fnrate_{1 \! \cap 2} = \Prob{(A_1 \! \cup A_2)' \Given B_1 \! \cap B_2}\,.
\end{equation}
By De Morgan's law, $(A_1 \! \cup A_2)' \equiv \SetIntersection[A_1'][A_2']$. Making this substitution results in
\begin{equation}
    \fnrate_{1 \! \cap 2} = \Prob{\SetIntersection[A_1'][A_2'] \Given B_1 \! \cap B_2}\,.
\end{equation}
Since $A_1$ and $A_2$ are independent events, by the rules of probability \begin{equation}
    \fnrate_{\SetIntersection[1][2]} = \Prob{A_1' \Given \SetIntersection[B_1][B_2]} \Prob{A_2' \Given \SetIntersection[B_1][B_2]}\,.
\end{equation}
Since $A_1$ is independent of $B_2$ and $A_2$ is independent of $B_1$, by the rules of probability
\begin{equation}
    \fnrate_{1 \! \cap 2} = \Prob{A_1' \Given B_1} \Prob{A_2' \Given B_2}\,.
\end{equation}
By definition, $\Prob{A_j' \Given B_j}$ is the false negative rate $\fnrate_j$. Making this substitution yields
\begin{equation}
    \fnrate_{1 \! \cap 2} = \fnrate_1 \fnrate_2\,.
\end{equation}
There are $\Card{\SetIntersection[\Set{S}[1]][\Set{S}[2]]}$ elements in $\SetIntersection[\Set{S}[1]][\Set{S}[2]]$, where each is an independent Bernoulli trial. Thus, there are expected to be
\begin{equation}
\label{eq:proof_union_fn_1}
    \Card{\Set{S}[1] \! \cap \Set{S}[2]} \fnrate_{1 \! \cap 2} = \Card{\Set{S}[1] \! \cap \Set{S}[2]} \fnrate_1 \fnrate_2
\end{equation}
false negatives in $\SetIntersection[\Set{S}[1]][\Set{S}[2]]$.

Suppose we randomly select an element from $\SetDiff[\Set{S}[1]][\Set{S}[2]]$. The probability that $\RV{X}$ is a negative in $\SetUnion[\ASet{S}[1]][\ASet{S}[2]]$ given that it is a positive in $\SetDiff[\Set{S}[1]][\Set{S}[2]]$ is given by
\begin{equation}
    \fnrate_{\SetIntersection[1][\SetComplement[2]]} = \Prob{\SetIntersection[A_1'][A_2'] \Given \SetIntersection[B_1][B_2']}\,.
\end{equation}
Since $A_1$ and $A_2$ are independent events, this may be rewritten as
\begin{equation}
    \fnrate_{\SetIntersection[1][\SetComplement[2]]} = \Prob{A_1' \Given B_1 \! \cap B_2'} \Prob{A_2' \Given B_1 \cap B_2'}\,.
\end{equation}
Since $A_1$ is independent of $B_2$ and $A_2$ is independent of $B_1$, this may be rewritten as
\begin{equation}
    \fnrate_{\SetIntersection[1][\SetComplement[2]]} = \Prob{A_1' \Given B_1} \Prob{A_2' \Given B_2'}\,.
\end{equation}
By definition, $\Prob{A_1' \Given B_1}$ is the false negative rate $\fnrate_1$ and $\Prob{A_2 \Given B_2'}$ is the false positive rate $\fprate_2$. Thus,
\begin{equation}
    \fnrate_{\SetIntersection[1][\SetComplement[2]]} = \fnrate_1 (1 - \fprate_2)\,.
\end{equation}
There are $\Card{\SetDiff[\Set{S}[1]][\Set{S}[2]]}$ elements in $\SetDiff[\Set{S}[1]][\Set{S}[2]]$, where each is an independent Bernoulli trial. Thus, there are expected to be
\begin{equation}
\label{eq:proof_union_fn_2}
    \Card{\SetDiff[\Set{S}[1]][\Set{S}[2]]} \fnrate_{\SetIntersection[1][\SetComplement[2]]} = \Card{\SetDiff[\Set{S}[1]][\Set{S}[2]]} \fnrate_1 (1 - \fprate_2)
\end{equation}
false negatives in $\SetDiff[\Set{S}[1]][\Set{S}[2]]$. A similar argument follows for $\SetDiff[\Set{S}[2]][\Set{S}[1]]$ where there are expected to be
\begin{equation}
\label{eq:proof_union_fn_3}
    \Card{\SetDiff[\Set{S}[2]][\Set{S}[1]]} \fnrate_2(1 - \fprate_2)
\end{equation}
false negatives.

The false negative rate is given by the ratio of the total expected number of false negatives given by \cref{eq:proof_union_fn_1,eq:proof_union_fn_2,eq:proof_union_fn_3} to the total number of possible false negatives $\Card{\Set{S}[1] \! \cup \Set{S}[2]}$, which is given by
\begin{equation}
\label{eq:proof_union_fn_4}
\begin{split}
    \fnrate =
        &\frac{\Card{\Set{S}[1] \! \setminus \Set{S}[2]}}{\Card{\Set{S}[1] \! \cup \Set{S}[2]}} \fnrate_1(1 - \fprate_2) + \frac{\Card{\Set{S}[2] \! \setminus \Set{S}[1]}}{\Card{\Set{S}[1] \! \cup \Set{S}[2]}} \fnrate_2(1 - \fprate_1) + \frac{\Card{\Set{S}[1] \! \cap \Set{S}[2]}}{\Card{\Set{S}[1] \! \cup \Set{S}[2]}} \fnrate_1 \fnrate_2\,.
\end{split}
\end{equation}
If we let
\begin{align}
    \alpha_1 = \frac{\Card{\Set{S}[1] \! \setminus \Set{S}[2]}}{\Card{\Set{S}[1] \! \cup \Set{S}[2]}}\; \text{and} \;
    \alpha_2 = \frac{\Card{\Set{S}[2] \! \setminus \Set{S}[1]}}{\Card{\Set{S}[1] \! \cup \Set{S}[2]}}\,,
\end{align}
then
\begin{equation}
    1 - \alpha_1 - \alpha_2 = \frac{\Card{\Set{S}[1] \! \cap \Set{S}[2]}}{\Card{\Set{S}[1] \! \cup \Set{S}[2]}}\,.
\end{equation}
Making these substitutions into \cref{eq:proof_union_fn_4} yields the result
\begin{equation}
\begin{split}
    \fnrate &= \alpha_1 \fnrate_1(1-\fprate_2) + \alpha_1 \fnrate_2(1-\fprate_1) +\\
         &\qquad (1 - \alpha_1 - \alpha_2) \fnrate_1 \fnrate_2\,.
\end{split}
\end{equation}
\end{proof}

\section{Sampling distribution of arbitrary functions}
\label{app:samp}
TODO: add generative model as an algorithm for approximate sets? Add C++ implementation of the model? Do some simulations to see how rapidly it converges to the normal?
TODO: feed in something like ppv function and see how well it matches the solution given in that one section. etc.

Suppose we have an objective function $\operatorname{f} \colon \PS{\Set{X}[1]} \times \cdots \times \PS{\Set{X}[q]} \mapsto \Set{Y}$, and we are interested in evaluating the loss when we replace one or more of the objective input sets with particular corresponding random approximate sets.
The result of this substitution, as previously described, is a probability distribution over $\Set{Y}$.

The probability distribution of random approximate sets are precisely given; therefore, we may estimate the distribution of any function of random approximate sets by generating the random approximate sets and applying the function of interest.

Consider the $m$-by-$q$ matrix where the $(i,j)$-th element is the random approximate set $\ASet{A}[i,j](\tprate_j,\fprate_j)$ such that they are all independely distributed and $\ASet{A}[i,j]$ for $i=1,\ldots, m$ are also identically distributed.
If we apply $\operatorname{g}$ to each row of the matrix,
\begin{equation}
\RV{Y_i} = \operatorname{g}\left(\ASet{A}[i,1],\ldots,\ASet{A}[i,q]\right)
\end{equation}
for $i=1,\ldots,m$, we generate $m$ i.i.d. random elements $\RV{Y_1},\ldots,\RV{Y_m}$.

If $\Set{Y}$ is a measure space (discrete or continuous), consider the random variable
\begin{equation}
\RV{\overline{Y}_m} = \frac{1}{m} \sum_{i=1}^{m} \RV{Y_i}\,.
\end{equation}
If $\RV{Y_1}$ has a well-defined mean and variance, then by the central limit theorem
\begin{equation}
\lim_{m \to \infty} \RV{\overline{Y}_m}
\end{equation}
converges in distribution to a normal with a mean $\Expect{\RV{Y_1}}$ and a variance $\Var{\RV{Y_1}} / m$.

A general approach to estimating $\RV{\overline{Y}_m}$ is given by generating a large sample of matrices and applying the function $\operatorname{g}$ to each to generate a large sample from $\RV{Y_1}$.

We provide an implementation of the generative model and a tool set that permits one to analyze various properties of the distribution of the function of interest.
}
\end{document}