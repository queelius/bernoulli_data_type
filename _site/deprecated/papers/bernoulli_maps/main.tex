%\documentclass[final,journal,compsoc,hidelinks]{article}
\documentclass[11pt,final,hidelinks]{article}
\usepackage{lmodern}
\usepackage[nohints]{minitoc}
\usepackage[margin=1in]{geometry}
\usepackage[english]{babel}
\usepackage{graphicx}
\graphicspath{img}
\usepackage[activate={true,nocompatibility},final,tracking=true,kerning=true,spacing=true,factor=1100,stretch=10,shrink=10]{microtype}
\microtypecontext{spacing=nonfrench}
\usepackage[noend]{algorithm2e}
\usepackage{caption}
\usepackage{bbm}
\usepackage{mathtools}
\usepackage{commath}
\usepackage{enumerate}
\usepackage{amsmath}
\usepackage{subfiles}
\usepackage{booktabs}
\usepackage[xindy,toc,acronyms,symbols]{glossaries}
\usepackage{latexsym}
\usepackage{amsthm}
\usepackage{amssymb}
\usepackage{pgfplots}
\usepackage{wasysym}
\usepackage{hyperref}
\usepackage{tikz}
\usepackage{tikzscale}
\usepackage[square,numbers]{natbib}
\bibliographystyle{plainnat}
\usepackage[utf8]{inputenc}
\usepackage[T1]{fontenc}
\usepackage{cleveref}
\usepackage[super]{nth}
\usepackage{siunitx}
\usepackage[section]{placeins}
\usepackage{amssymb}
\usepackage{minted}
\usepackage{multicol}
\usepackage{stmaryrd}

\newcommand{\COMMENT}[1]{}

\SetKwFunction{Loss}{loss}
\SetKwFunction{HasKey}{has\_key}
\SetKwData{Nothing}{nothing}

\newcommand{\mathlg}[1]{\mathlarger{\mathlarger{\mathlarger{#1}}}}

\SetKwFunction{Match}{match}
\SetKwFunction{MakeSHM}{construct\_shm}
\SetKwData{SHM}{SHM}

\SetKwFunction{GenerateSeeds}{generate\_hash\_seeds}

\SetKwFunction{MakeSingularHashMap}{construct\_singular\_hash\_map}

\SetKwData{found}{found}
\SetKwData{truecode}{true\_code}
\SetKwData{code}{test\_code}
\SetKwData{Code}{code}
\SetKwData{Success}{success}


\SetKwFunction{trunc}{tr}
\SetKwFunction{Encode}{encode}
\SetKwFunction{Decode}{decode}



\usepackage{functionnotation}
\usepackage[section]{envnotation}
\usepackage{setnotation}
\usepackage{relationnotation}
\usepackage{approxsetnotation}
\usepackage{approxrelationnotation}
\usepackage{algorithmnotation}

\title
{
    The algebra of the \emph{random approximate map} model
}
\author
{
    Alexander Towell\\
    \texttt{atowell@siue.edu}
}
\date{}

%\loadglsentries{gloss}

\newcommand{\amapsto}[2]{\,\ATOverUnder{\mapsto}[#1][\text{\raisebox{3pt}{$#2$}}]\,}

\begin{document}
\maketitle
\begin{abstract}
We define the semantics of the \emph{approximate map}, which is a \emph{probabilistic} map that \emph{approximates} another map with a false positive rate $\fprate$ and false negative rate $\fnrate$. We derive several properties that follow from these semantics, such as the expected \emph{precision} in information retrieval. Following from this, we generalize to approximate relations and compositions. Finally, we demonstrate an application especially relevant to \emph{Encrypted Search}, rank-ordered search, which generates \emph{approximate result sets} with approximate values and show how the relational model may be used to support different ranking methods. Finally, we extend this information retrieval model to a fuzzy set-theoretic query model.
\end{abstract}

\tableofcontents
%\listoffigures
%\listofalgorithms

\subfile{sections/intro}
\subfile{sections/approxmaps}
\subfile{sections/approx_maps}
\subfile{sections/composition}
\subfile{sections/special_functions}
\subfile{sections/disc_fun}
%\subfile{sections/adt}
\subfile{sections/singular_hash_map}
%\subfile{sections/random_algebraic}
%\subfile{sections/notes}
%\subfile{sections/rank_ordered_search}
%\subfile{sections/appendix}
\printglossary
\bibliography{references}
\end{document}
