% Options for packages loaded elsewhere
\PassOptionsToPackage{unicode}{hyperref}
\PassOptionsToPackage{hyphens}{url}
%
\documentclass[
]{article}
\usepackage{lmodern}
\usepackage{amssymb,amsmath}
\usepackage{ifxetex,ifluatex}
\ifnum 0\ifxetex 1\fi\ifluatex 1\fi=0 % if pdftex
  \usepackage[T1]{fontenc}
  \usepackage[utf8]{inputenc}
  \usepackage{textcomp} % provide euro and other symbols
\else % if luatex or xetex
  \usepackage{unicode-math}
  \defaultfontfeatures{Scale=MatchLowercase}
  \defaultfontfeatures[\rmfamily]{Ligatures=TeX,Scale=1}
\fi
% Use upquote if available, for straight quotes in verbatim environments
\IfFileExists{upquote.sty}{\usepackage{upquote}}{}
\IfFileExists{microtype.sty}{% use microtype if available
  \usepackage[]{microtype}
  \UseMicrotypeSet[protrusion]{basicmath} % disable protrusion for tt fonts
}{}
\makeatletter
\@ifundefined{KOMAClassName}{% if non-KOMA class
  \IfFileExists{parskip.sty}{%
    \usepackage{parskip}
  }{% else
    \setlength{\parindent}{0pt}
    \setlength{\parskip}{6pt plus 2pt minus 1pt}}
}{% if KOMA class
  \KOMAoptions{parskip=half}}
\makeatother
\usepackage{xcolor}
\IfFileExists{xurl.sty}{\usepackage{xurl}}{} % add URL line breaks if available
\IfFileExists{bookmark.sty}{\usepackage{bookmark}}{\usepackage{hyperref}}
\hypersetup{
  pdftitle={A Boolean Algebra Over Trapdoors},
  hidelinks,
  pdfcreator={LaTeX via pandoc}}
\urlstyle{same} % disable monospaced font for URLs
\usepackage[margin=1in]{geometry}
\usepackage{color}
\usepackage{fancyvrb}
\newcommand{\VerbBar}{|}
\newcommand{\VERB}{\Verb[commandchars=\\\{\}]}
\DefineVerbatimEnvironment{Highlighting}{Verbatim}{commandchars=\\\{\}}
% Add ',fontsize=\small' for more characters per line
\usepackage{framed}
\definecolor{shadecolor}{RGB}{248,248,248}
\newenvironment{Shaded}{\begin{snugshade}}{\end{snugshade}}
\newcommand{\AlertTok}[1]{\textcolor[rgb]{0.94,0.16,0.16}{#1}}
\newcommand{\AnnotationTok}[1]{\textcolor[rgb]{0.56,0.35,0.01}{\textbf{\textit{#1}}}}
\newcommand{\AttributeTok}[1]{\textcolor[rgb]{0.77,0.63,0.00}{#1}}
\newcommand{\BaseNTok}[1]{\textcolor[rgb]{0.00,0.00,0.81}{#1}}
\newcommand{\BuiltInTok}[1]{#1}
\newcommand{\CharTok}[1]{\textcolor[rgb]{0.31,0.60,0.02}{#1}}
\newcommand{\CommentTok}[1]{\textcolor[rgb]{0.56,0.35,0.01}{\textit{#1}}}
\newcommand{\CommentVarTok}[1]{\textcolor[rgb]{0.56,0.35,0.01}{\textbf{\textit{#1}}}}
\newcommand{\ConstantTok}[1]{\textcolor[rgb]{0.00,0.00,0.00}{#1}}
\newcommand{\ControlFlowTok}[1]{\textcolor[rgb]{0.13,0.29,0.53}{\textbf{#1}}}
\newcommand{\DataTypeTok}[1]{\textcolor[rgb]{0.13,0.29,0.53}{#1}}
\newcommand{\DecValTok}[1]{\textcolor[rgb]{0.00,0.00,0.81}{#1}}
\newcommand{\DocumentationTok}[1]{\textcolor[rgb]{0.56,0.35,0.01}{\textbf{\textit{#1}}}}
\newcommand{\ErrorTok}[1]{\textcolor[rgb]{0.64,0.00,0.00}{\textbf{#1}}}
\newcommand{\ExtensionTok}[1]{#1}
\newcommand{\FloatTok}[1]{\textcolor[rgb]{0.00,0.00,0.81}{#1}}
\newcommand{\FunctionTok}[1]{\textcolor[rgb]{0.00,0.00,0.00}{#1}}
\newcommand{\ImportTok}[1]{#1}
\newcommand{\InformationTok}[1]{\textcolor[rgb]{0.56,0.35,0.01}{\textbf{\textit{#1}}}}
\newcommand{\KeywordTok}[1]{\textcolor[rgb]{0.13,0.29,0.53}{\textbf{#1}}}
\newcommand{\NormalTok}[1]{#1}
\newcommand{\OperatorTok}[1]{\textcolor[rgb]{0.81,0.36,0.00}{\textbf{#1}}}
\newcommand{\OtherTok}[1]{\textcolor[rgb]{0.56,0.35,0.01}{#1}}
\newcommand{\PreprocessorTok}[1]{\textcolor[rgb]{0.56,0.35,0.01}{\textit{#1}}}
\newcommand{\RegionMarkerTok}[1]{#1}
\newcommand{\SpecialCharTok}[1]{\textcolor[rgb]{0.00,0.00,0.00}{#1}}
\newcommand{\SpecialStringTok}[1]{\textcolor[rgb]{0.31,0.60,0.02}{#1}}
\newcommand{\StringTok}[1]{\textcolor[rgb]{0.31,0.60,0.02}{#1}}
\newcommand{\VariableTok}[1]{\textcolor[rgb]{0.00,0.00,0.00}{#1}}
\newcommand{\VerbatimStringTok}[1]{\textcolor[rgb]{0.31,0.60,0.02}{#1}}
\newcommand{\WarningTok}[1]{\textcolor[rgb]{0.56,0.35,0.01}{\textbf{\textit{#1}}}}
\usepackage{graphicx}
\makeatletter
\def\maxwidth{\ifdim\Gin@nat@width>\linewidth\linewidth\else\Gin@nat@width\fi}
\def\maxheight{\ifdim\Gin@nat@height>\textheight\textheight\else\Gin@nat@height\fi}
\makeatother
% Scale images if necessary, so that they will not overflow the page
% margins by default, and it is still possible to overwrite the defaults
% using explicit options in \includegraphics[width, height, ...]{}
\setkeys{Gin}{width=\maxwidth,height=\maxheight,keepaspectratio}
% Set default figure placement to htbp
\makeatletter
\def\fps@figure{htbp}
\makeatother
\setlength{\emergencystretch}{3em} % prevent overfull lines
\providecommand{\tightlist}{%
  \setlength{\itemsep}{0pt}\setlength{\parskip}{0pt}}
\setcounter{secnumdepth}{5}
\usepackage{amsmath}
\usepackage{amssymb}
\usepackage{amsthm}

\title{A Boolean Algebra Over Trapdoors}
\author{}
\date{\vspace{-2.5em}2023-06-17}

\begin{document}
\maketitle
\begin{abstract}
This paper introduces a Boolean algebra framework over trapdoors,
establishing a novel approach to cryptographic operations within a
Boolean algebraic structure. The core of the framework is the Boolean
algebra \(A := (\mathcal{P}(X^*), \land, \lor, \neg, \emptyset, X^*)\),
with \(\mathcal{P}\) representing the powerset and \(X^*\) the free
semigroup on an alphabet \(X\). A key feature of this study is the
homomorphism \(F : A \mapsto B\) from \(A\) to a Boolean algebra
\(B := (\{0,1\}^n, \&, |, ~, 0^n, 1^n)\) of \(n\)-bit strings, achieved
through a cryptographic hash function. This homomorphism introduces a
secret \(s\) into its operation, embedding security within the algebraic
structure and rendering \(F\) one-way. Our exploration highlights the
cryptographic utility of this framework, especially in terms of
collision probability and resistance to reverse engineering, offering a
foundational basis for secure cryptographic operations leveraging
Boolean algebra.
\end{abstract}

{
\setcounter{tocdepth}{2}
\tableofcontents
}
\newtheorem{theorem}{Theorem}

Consider the Boolean algebra

\[
    A := (\mathcal{P}(X^*), \land, \lor, \neg, \emptyset, X^*)
\]

where \(\mathcal{P}\) is the powerset, \(X\) is the alphabet, and
\(X^*\) is the free semigroup on \(X\) which is closed under
concatenation,

\[
    \# : X^* \mapsto X^* \mapsto X^*.
\]

For example, if \[
    X = \{a,b\}
\] then \[
    X^* = \{\epsilon, a, b, aa, ab, ba, bb, aaa, aab, \ldots \}
\] and \(\mathcal{P}(X^*)\) is the power set of \(X^*\), \[
    \mathcal{P}(X^*) = {\emptyset, \epsilon, \{a\}, \{b\},
             \{aa\}, \{a,aa\}, \{a,bb\}, \ldots }.
\] Consider the Boolean algebra \[
    B := (\{0,1\}^n, \&, |, ~, 0^n, 1^n)
\] and suppose we have a homomorphism \[
    F : A \mapsto B
\] defined in the following way. First, we have a cryptographic hash
function \[
    \rm{hash} : X* \mapsto {0,1}^n
\] that a priori uniformly distributes over \(\{0,1\}^n\), i.e., each
\(X^*\) maps to any element in the \(\{0,1\}^n\) with probability
\(2^{-n}\).

Then, homomorphism \(F\) maps strings in \(X^*\) to bit strings in
\(\{0,1\}^n\) by applying the hash function to the input concatenated
with a secret \(s\), \[
    F a := \rm{hash}(a s).
\]

Note \#1: Later, we generalize this to mapping each \(a\) in \(X^*\) to
multiple elements in \(\{0,1\}^n\) proportional to \(1/P[a]\).

Observe that \(F\) is one-way, i.e., there is no homomorphism \(G\) such
that \[
    F G B = A.
\]

\begin{theorem}
The morphism F defined as
\begin{align*}
    X^*        &:= hash(a \# s)\\
    \rm{and}        &:= \&\\
    \rm{or}         &:= |\\
    \rm{complement} &:= \sim\\
    \{\}       &:= 0^n\\
    X^*        &:= 1^n.
\end{align*}
is a homomorphism.
\end{theorem}

\begin{proof}
The proof is trivial so we omit it.
\end{proof}

Since multiple elements in \(X^*\) map to the same element in
\(\{0,1\}^n\), it is a homomorphism rather than an isomorphism.

What is the probability that two unique elements in \(X^*\) map to the
same element in \(\{0,1\}^n\)? That is to say, what is the probability
of collision? Since \(F\) uniformly distributes over \(\{0,1\}^n\), it
is just \[
    Pr\{\text{$x$ and $y$ collide}\} = 2^(-n).
\] By the law of probability, therefore, the probability that they do
not collide is just \[
    Pr\{\text{$x$ and $y$ do not collide}\} = 1 - 2^(-n).
\] Next, we define relations on sets. Set membership relation has a
characteristic function

\begin{Shaded}
\begin{Highlighting}[]
\NormalTok{   in : X {-}\textgreater{} }\DecValTok{2}\NormalTok{\^{}X {-}\textgreater{} }\DataTypeTok{bool}
\end{Highlighting}
\end{Shaded}

which we define as

\begin{Shaded}
\begin{Highlighting}[]
\NormalTok{   F in a b := a \textbackslash{}\& b == a.}
\end{Highlighting}
\end{Shaded}

The subset relation has a predicate

\begin{Shaded}
\begin{Highlighting}[]
\NormalTok{   subset : }\DecValTok{2}\NormalTok{\^{}X {-}\textgreater{} }\DecValTok{2}\NormalTok{\^{}X {-}\textgreater{} }\DataTypeTok{bool}
\end{Highlighting}
\end{Shaded}

which we define as

\begin{Shaded}
\begin{Highlighting}[]
\NormalTok{   F subset a b := a \textbackslash{}\& b == a,}
\end{Highlighting}
\end{Shaded}

just as with the characteristic function, although they have different
probabilistic features.

If \(X = \{a,b,c\}\), then
\(2^X = {{},{a},{b},{c},{a,b},{a,c},{b,c},{a,b,c}}\).

A Boolean index over \(X\) is a Boolean algebra over \(2^X\) with
\({}=0\) and \(X=1\) with the normal set operations. This is what a lot
of prior work was over.

Note that a type that models
\texttt{power\_set\textless{}trapdoor\textless{}X\textgreater{}\textgreater{}}
is one in which given a value \texttt{A} of this type, each element
\texttt{a} in \texttt{A} is a
\texttt{trapdoor\textless{}X\textgreater{}} can be independently
observed. This makes it possible to operate on \texttt{A} as a normal
set, with the exception that the mapping the trapdoors to values may not
be obvious (although given a history, or a set of sets, frequency
analysis or correlation analysis may reveal quite a bit).

\begin{Shaded}
\begin{Highlighting}[]
\KeywordTok{template}\NormalTok{ \textless{}}\KeywordTok{typename}\NormalTok{ X, }\DataTypeTok{size\_t}\NormalTok{ N\textgreater{}}
\KeywordTok{struct}\NormalTok{ trapdoor\_boolean\_algebra}
\NormalTok{\{}
    \KeywordTok{using} \DataTypeTok{value\_type}\NormalTok{ = X;}

\NormalTok{    trapdoor\_boolean\_algebra() :}
\NormalTok{        value\_hash(}\DecValTok{0}\NormalTok{),}
\NormalTok{        key\_hash(}\DecValTok{0}\NormalTok{)}
\NormalTok{    \{}
        \CommentTok{// makes the empty set}
\NormalTok{    \}}

\NormalTok{    trapdoor\_boolean\_algebra(trapdoor\_boolean\_algebra }\AttributeTok{const}\NormalTok{ \&) = }\ControlFlowTok{default}\NormalTok{;}

\NormalTok{    array\textless{}}\DataTypeTok{char}\NormalTok{, N\textgreater{} value\_hash;}
\NormalTok{    array\textless{}}\DataTypeTok{char}\NormalTok{, }\DecValTok{4}\NormalTok{\textgreater{} key\_hash;}
\NormalTok{\};}
\end{Highlighting}
\end{Shaded}

\begin{Shaded}
\begin{Highlighting}[]
\KeywordTok{template}\NormalTok{ \textless{}}\KeywordTok{typename}\NormalTok{ X, }\DataTypeTok{size\_t}\NormalTok{ N\textgreater{}}
\KeywordTok{auto}\NormalTok{ make\_empty\_trapdoor\_set()}
\NormalTok{\{}
    \ControlFlowTok{return}\NormalTok{ trapdoor\_boolean\_algebra\textless{}X,N\textgreater{}();}
\NormalTok{\}}
\end{Highlighting}
\end{Shaded}

\begin{Shaded}
\begin{Highlighting}[]
\CommentTok{/**}
\CommentTok{The disjoint union operation is a partial function that is only defined}
\CommentTok{when the argument sets are disjoint (it is a dependent type). If they are}
\CommentTok{not disjoint, the operation has undefined behavior.}
\CommentTok{ */}
\KeywordTok{template}\NormalTok{ \textless{}}\KeywordTok{typename}\NormalTok{ X, }\DataTypeTok{size\_t}\NormalTok{ N\textgreater{}}
\KeywordTok{auto} \KeywordTok{operator}\NormalTok{+(}
\NormalTok{    trapdoor\_boolean\_algebra\textless{}X,N\textgreater{} }\AttributeTok{const}\NormalTok{ \& x,}
\NormalTok{    trapdoor\_boolean\_algebra\textless{}X,N\textgreater{} }\AttributeTok{const}\NormalTok{ \& y)}
\NormalTok{\{}
    \ControlFlowTok{if}\NormalTok{ (x.key\_hash != y.key\_hash)}
        \ControlFlowTok{throw}\NormalTok{ invalid\_argument(}\StringTok{"secret key mismatch"}\NormalTok{);}

    \ControlFlowTok{return}\NormalTok{ trapdoor\_boolean\_algebra\textless{}X\textgreater{}(}
\NormalTok{        x.value\_hash | y.value\_hash,}
\NormalTok{        x.key\_hash);}
\NormalTok{\}}
\end{Highlighting}
\end{Shaded}

\begin{Shaded}
\begin{Highlighting}[]
\KeywordTok{template}\NormalTok{ \textless{}}\KeywordTok{typename}\NormalTok{ X, }\DataTypeTok{size\_t}\NormalTok{ N\textgreater{}}
\KeywordTok{auto} \KeywordTok{operator}\NormalTok{!(}
\NormalTok{    trapdoor\_boolean\_algebra\textless{}X,N\textgreater{} }\AttributeTok{const}\NormalTok{ \& x)}
\NormalTok{\{}
    \ControlFlowTok{return}\NormalTok{ trapdoor\_boolean\_algebra\textless{}X\textgreater{}(}
\NormalTok{        \textasciitilde{}x.value\_hash,}
\NormalTok{        x.key\_hash);}
\NormalTok{\}}
\end{Highlighting}
\end{Shaded}

\begin{Shaded}
\begin{Highlighting}[]
\KeywordTok{template}\NormalTok{ \textless{}}\KeywordTok{typename}\NormalTok{ X, }\DataTypeTok{size\_t}\NormalTok{ N\textgreater{}}
\KeywordTok{auto} \KeywordTok{operator}\NormalTok{*(}
\NormalTok{    trapdoor\_boolean\_algebra\textless{}X,N\textgreater{} }\AttributeTok{const}\NormalTok{ \& x,}
\NormalTok{    trapdoor\_boolean\_algebra\textless{}X,N\textgreater{} }\AttributeTok{const}\NormalTok{ \& y)}
\NormalTok{\{}
    \ControlFlowTok{if}\NormalTok{ (x.key\_hash != y.key\_hash)}
        \ControlFlowTok{throw}\NormalTok{ invalid\_argument(}\StringTok{"secret key mismatch"}\NormalTok{);}

    \ControlFlowTok{return}\NormalTok{ trapdoor\_boolean\_algebra\textless{}X\textgreater{}(}
\NormalTok{        x.value\_hash \& y.value\_hash,}
\NormalTok{        x.key\_hash);}
\NormalTok{\}}
\end{Highlighting}
\end{Shaded}

\begin{Shaded}
\begin{Highlighting}[]
\KeywordTok{template}\NormalTok{ \textless{}}\KeywordTok{typename}\NormalTok{ X, }\KeywordTok{typename}\NormalTok{ Y, }\DataTypeTok{size\_t}\NormalTok{ N\textgreater{}}
\KeywordTok{auto}\NormalTok{ disjoint\_union(}
\NormalTok{    trapdoor\_boolean\_algebra\textless{}X,N\textgreater{} }\AttributeTok{const}\NormalTok{ \& x,}
\NormalTok{    trapdoor\_boolean\_algebra\textless{}Y,N\textgreater{} }\AttributeTok{const}\NormalTok{ \& y)}
\NormalTok{\{}
    \ControlFlowTok{if}\NormalTok{ (x.key\_hash != y.key\_hash)}
        \ControlFlowTok{throw}\NormalTok{ invalid\_argument(}\StringTok{"secret key mismatch"}\NormalTok{);}

    \ControlFlowTok{return}\NormalTok{ trapdoor\_boolean\_algebra\textless{}variant\textless{}X,Y\textgreater{}\textgreater{}(}
\NormalTok{        x.value\_hash | y.value\_hash,}
\NormalTok{        x.key\_hash);}
\NormalTok{\}}
\end{Highlighting}
\end{Shaded}

\begin{Shaded}
\begin{Highlighting}[]
\CommentTok{// the bernoulli\textless{}bool\textgreater{} stores the log{-}probability of the value being incorrect}
\KeywordTok{template}\NormalTok{ \textless{}}\KeywordTok{typename}\NormalTok{ X, }\DataTypeTok{size\_t}\NormalTok{ N\textgreater{}}
\NormalTok{bernoulli\textless{}}\DataTypeTok{bool}\NormalTok{\textgreater{} empty(trapdoor\_boolean\_algebra\textless{}X,N\textgreater{} }\AttributeTok{const}\NormalTok{ \& xs)}
\NormalTok{\{}
    \KeywordTok{auto}\NormalTok{ b = }\BuiltInTok{std::}\NormalTok{all\_of(xs.begin(),xs.end(),[](}\DataTypeTok{char}\NormalTok{ x) \{ }\ControlFlowTok{return}\NormalTok{ x == }\DecValTok{0}\NormalTok{; \});}
    \ControlFlowTok{return}\NormalTok{ bernoulli\textless{}}\DataTypeTok{bool}\NormalTok{\textgreater{}\{b,}\FloatTok{0.5}\NormalTok{\};}
\NormalTok{\}}

\KeywordTok{template}\NormalTok{ \textless{}}\KeywordTok{typename}\NormalTok{ X, }\DataTypeTok{size\_t}\NormalTok{ N\textgreater{}}
\NormalTok{bernoulli\textless{}}\DataTypeTok{bool}\NormalTok{\textgreater{} contains(}
\NormalTok{    trapdoor\textless{}X,N\textgreater{} }\AttributeTok{const}\NormalTok{ \& x,}
\NormalTok{    trapdoor\_boolean\_algebra\textless{}X,N\textgreater{} }\AttributeTok{const}\NormalTok{ \& xs)}
\NormalTok{\{}
    \ControlFlowTok{if}\NormalTok{ (x.key\_hash != xs.key\_hash)}
        \ControlFlowTok{throw} \BuiltInTok{std::}\NormalTok{invalid\_argment(}\StringTok{"secret key mismatch"}\NormalTok{);}
    \KeywordTok{auto}\NormalTok{ b = }\BuiltInTok{std::}\NormalTok{all\_of(xs.begin(),xs.end(),[](}\DataTypeTok{char}\NormalTok{ x) \{ }\ControlFlowTok{return}\NormalTok{ x == }\DecValTok{0}\NormalTok{; \});}
    \ControlFlowTok{return}\NormalTok{ bernoulli\textless{}}\DataTypeTok{bool}\NormalTok{, }\DecValTok{1}\NormalTok{\textgreater{}\{b, }\FloatTok{.5}\NormalTok{\};}
\NormalTok{\}}

\KeywordTok{template}\NormalTok{ \textless{}}\KeywordTok{typename}\NormalTok{ X\textgreater{}}
\NormalTok{approximate\_bool }\KeywordTok{operator}\NormalTok{\textless{}=(}
\NormalTok{    trapdoor\_boolean\_algebra\textless{}X\textgreater{} }\AttributeTok{const}\NormalTok{ \& x,}
\NormalTok{    trapdoor\_boolean\_algebra\textless{}X\textgreater{} }\AttributeTok{const}\NormalTok{ \& y)}
\NormalTok{\{}
    \KeywordTok{auto}\NormalTok{ b = }\BuiltInTok{std::}\NormalTok{all\_of(xs.begin(),xs.end(),[](}\DataTypeTok{char}\NormalTok{ x) \{ }\ControlFlowTok{return}\NormalTok{ x == }\DecValTok{0}\NormalTok{; \});}
    \ControlFlowTok{return}\NormalTok{ approximate\_bool\{b, }\FloatTok{.5}\NormalTok{\};}
\NormalTok{\}}

\KeywordTok{template}\NormalTok{ \textless{}}\KeywordTok{typename}\NormalTok{ X\textgreater{}}
\NormalTok{approximate\_bool }\KeywordTok{operator}\NormalTok{==(}
\NormalTok{    trapdoor\_boolean\_algebra\textless{}X\textgreater{} }\AttributeTok{const}\NormalTok{ \& x}
\NormalTok{    trapdoor\_boolean\_algebra\textless{}X\textgreater{} }\AttributeTok{const}\NormalTok{ \& y)}
\NormalTok{\{}
    \KeywordTok{auto}\NormalTok{ b = }\BuiltInTok{std::}\NormalTok{all\_of(xs.begin(),xs.end(),[](}\DataTypeTok{char}\NormalTok{ x) \{ }\ControlFlowTok{return}\NormalTok{ x == }\DecValTok{0}\NormalTok{; \});}
    \ControlFlowTok{return}\NormalTok{ approximate\_bool\{b, }\FloatTok{.5}\NormalTok{\};}
\NormalTok{\}}


\KeywordTok{template}\NormalTok{ \textless{}}\KeywordTok{typename}\NormalTok{ X, }\DataTypeTok{size\_t}\NormalTok{ N\textgreater{}}
\KeywordTok{auto}\NormalTok{ hash(trapdoor\_boolean\_algebra\textless{}X,N\textgreater{} }\AttributeTok{const}\NormalTok{ \& x)}
\NormalTok{\{}
    \ControlFlowTok{return}\NormalTok{ x.value\_hash \^{} x.key\_hash \^{} hash(}\KeywordTok{typeid}\NormalTok{(X))}
\NormalTok{\}}
\end{Highlighting}
\end{Shaded}


\end{document}
